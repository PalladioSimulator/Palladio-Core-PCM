\section{Entwurf}
\label{sec:entwurf}
//TODO: Einleitung zum Kapitel
Die praktische Umsetzung der im letzten Kapitel diskutierten Modelle ist wie folgt aufgegliedert.

\subsection{Anforderungen}
\label{sec:entwurf:anf}

Das Gebiet der Simulation von Komponentenarchitekturen ist z.Z. praktisch relativ unerforscht. Dieses Framework bildet den Versuch der Umsetzung einer Simulationsumgebung, ohne auf bereits bew�hrte Methoden auf diesem Gebiet zur�ckgreifen zu k�nnnen. Hierbei ergeben sich zwei zwingende funktionale Anforderungen.

\subsection{Bereitstellung von Basisfunktionalit�ten}
\label{sec:entwurf:anf:basis}
Das Framework soll in allen im Folgenden erl�uterten Bereichen zusammen lauff�hige Basisfunktionalit�ten bieten. Dies kann zum einen f�r den unver�nderten Gebrauch von n�ten sein, wobei dann ohne weitere Implementierungen Simulationen auf dem Framework ausgef�hrt werden k�nnen. Wichtiger in diesem Zusammenhang erscheint jedoch die gezielte Erweiterung an bestimmten Stellen des Framework, wobei andere Bereiche unver�ndert genutzt werden k�nnen. Die zu unterst�tzenden Bereiche der Simulationsumgebung lassen sich grob in drei Teile aufspalten. 

\begin{itemize}

\listitem{Modellierung einer Komponentenarchitektur}
Dieser Teil sorgt f�r die Umsetzung des in Kapitel \ref{sec:modell} vorgestellten Komponentenmodells. Hierbei gilt es, den Aufbau eines Architekturmodells in einem f�r die Simulation nutzbaren Format zu erm�glichen.
	
\listitem{Simulation von Anfragen}
Im zweiten Teil geht es um die Simulation von Anfragen an das mit dem ersten Teil aufgebauten Modell. Hierzu soll das Simulationsmodell aus Kapitel \ref{sec:modell} umgesetzt werden. 

\listitem{Erfassung von Simulationsdaten}
Hier sollen nun die w�hrend der Simulation enstandenen Daten gesammelt werden. Bei der Implementierung der Basisfunktionalit�t gen�gt es, Daten sp�ter auswerten zu k�nnen. Um jedoch auch w�hrend der Simulation Analysen vornehmen zu k�nnen, soll dieser Teil die M�glichkeit offen lassen, Daten auch direkt auswerten zu k�nnen.
\end{itemize}
	
\subsection{M�glichkeiten zum Austausch und zur Erweiterbarkeit}
\label{sec:entwurf:anf:erweit}

Aufgrund oben angesprochener Problematik, ist beim Entwurf des Frameworks darauf zu achten, das es m�glichst modular aufgebaut ist. Das bedeutet, es m�ssen sich alle gr��eren Bestandteile des Frameworks austauschen lassen, ohne die Funktionalit�t anderer Bereiche ebenfalls �ndern zu m�ssen. So ist es beispielsweise w�nschenswert, den Teil f�r die Modulierung auf ein anderes Komponentenmodell umstellen zu k�nnen und mit diesem die Funktionalit�t des Simulations- und des Auswertungsteil verwenden zu k�nnen.\par
Weiterhin sollen die Basisimplementierungen nicht nur ausschlie�lich austauschbare Referenzimplementierungen sein, sondern die M�glichkeit der Erweiterung offen lassen.  

\subsection{Architektur des Frameworks}
\label{sec:entwurf:grob}

\subsection{Feinentwurf}
\label{sec:entwurf:fein}
