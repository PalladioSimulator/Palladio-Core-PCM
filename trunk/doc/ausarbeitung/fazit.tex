\section{Fazit}
\label{sec:anhang}

Das letzte Kapitel dieser Ausarbeitung fasst die Ergebnise des Projektes zusammen. Weiterhin wird ein kurzer �berblick �ber den Verlauf des Projektes in Hinblick auf die anf�nglich aufgestellten Ziele und die zum Erreichen dieser Ziele zu l�senden Probleme gegeben. Ein zum Schluss folgender Ausblick soll Anregungen zu m�glichen Anwendungen und Erweiterungsm�glichkeiten des Frameworks im Rahmen von neuen Projekten geben.
\par
Das {\em Framework zur Simulation komponentenbasierter Softwarearchitekturen} bildet eine gute Grundlage zum einfachen Simulieren von Komponentenarchitekturen mit statischen Zeitverbauchern. Das Erweitern des Frameworks um dynamische Zeitverbaucher ist leicht m�glich, ohne Implementierungen des Frameworks ver�ndern zu m�ssen. Weiterhin sieht das Design des Frameworks Eingangs erl�uterte Erweiterungsm�glichkeiten vor, wodurch sich das Framework auch zur Integration in eine fertige Anwendung eignet.

\subsection{Projektverlauf}
\label{sec:anhang:verlauf}

Der urspr�ngliche Plan des Projektverlaufes, wie er im Proposal (\cite{lit:proposal}) und in der Einleitung dieser Ausarbeitung (Kapitel \ref{sec:einleitung:ziele}) vorgestellt wurde, sah die Einteilung des Projektes in drei Entwicklungsinkremente vor. Das erste dieser Inkremente bildet mit der Erstellung der Basisfunktionalit�t den Hauptteil des Projektes. Das zweite Inkrement sollte diese Funktionalit�t um dynamische Aspekte erweitern. Im letzten optionale Inkrement war die Erstellung einer grafischen Auswertung vorgesehen.
\par
Bei der Umsetzung des ersten Inkrementes zeigte sich schnell, dass die strikte Trennung zwischen dem ersten und dem zweiten Inkrement nicht sinnvoll umsetzbar war. So w�rde bereits der Entwurfs des Frameworks ohne Betrachtung von dynamischen Aspekten Designschw�chen beinhalten, die bei der Umsetzung des zweiten Inkrements einen beinahe neuen Entwurf erzwungen h�tten. Aufgrund dieser Tatsache wurden bereits bei der Erstellung der Basisfunktionalit�t auf die dynamischen Aspekte eingegangen. Dies hatte zur Folge, dass die beiden Inkremente ineinander �bergingen. Als Ergebnis dieser beiden Inkremente enstand das Framework in seinem jetzigen Stand.
\par
Das in der das Projekt begleitenden Abteilung bisher ungel�ste Problem der Modellierung mehrerer Instanzen von Komponenten (siehe Kapitel \ref{sec:anw:grenzen}) und den damit verbundenen Problemen bei der Modellierung dynamischer Zeitverbraucher, gab Anlass, das Angebot des Frameworks an Zeitverbrauchern nicht um dynamische Elemente zu erweitern.
\par
Probleme und damit verbundene Verz�gerungen bei der Verwendung der von Palladio angebotenen Bibliothen haben dazu gef�hrt, dass nach Absprache mit der Abteilung das dritte optinale Inkrement fallen gelassen wurde.

\subsection{Ausblick}
\label{sec:anhang:ausblick}

Das Framework in seiner jetzigen Form bietet viele Erweiterungsm�glichkeit. So bildet beispielsweise die Entwicklung einer geeigneten Darstellung der Komponentenarchitektur und eine damit m�gliche Visualisierung der Simulation die Grundidee f�r ein das Framework nutzendes Tool. Dieses Tool k�nnte dann weiterhin als Plugin f�r einen Editor dienen, welcher den Aufbau des Komponentenmodells �bernimmt.
\par
Eine andere Art der Erweiterung liegt in der Umsetzung neuer Modelle der Zeitverbraucher. So kann das Framework im Laufe der Entwicklung der Forschung mit immer neuen dynamsichen Zeitverbrauchern betrieben werden. Auch gr��ere �nderungen des Modells sind unter Verwendung des Frameworks m�glich. So kann bespielsweise die gesamte Infrastruktur zu Simulation erhalten bleiben, w�hrend der Aufbau und interne Repr�sentation des Modells der Komponentenarchitektur sich v�llig �ndert. Ein Beispiel hierf�r ist in Kapitel \ref{sec:anw:grenzen} beschrieben.
