% $Log$
% Revision 1.4  2006/01/25 14:43:00  kelsaka
% - titelfolie: datum
% - mehrfache-delegation bei assembly konnektoren
%
% Revision 1.3  2006/01/22 13:09:02  kelsaka
% - anotation --> annotation
%
% Revision 1.2  2006/01/15 13:43:23  kelsaka
% - allokation, assembly, deployment, ...
%
% Revision 1.1  2006/01/05 16:46:14  kelsaka
% - initial creation
%


\documentclass[12pt]{scrartcl}

\usepackage[a4paper,left=3.0cm,right=2.5cm,top=3.0cm,bottom=2.5cm,headheight=15pt,headsep=1.5cm,footskip=1cm]{geometry}

%---- Sonderzeichen-------%
\usepackage {ngerman}
\usepackage{umlaut}
%---- Codierung----%
\usepackage[latin1]{inputenc}	% f�r Unix und Windows
%\usepackage[applemac]{inputenc}	% f�r MAC

%----- Mathematischer Zeichenvorrat---%
\usepackage{amsmath}
\usepackage{amssymb}

%----- Graphik ------%
\usepackage{graphicx}
%\usepackage[pdftex]{graphics}
\usepackage{makeidx}

\usepackage{enumerate}

% fuer die aktuelle Zeit
\usepackage{scrtime}



%\usepackage{hyperref}
%\usepackage{lastpage}       % Seitennummer der letzten Seite auslesen

\usepackage[colorlinks=true, pdfstartview=FitV, linkcolor=black,
citecolor=black, urlcolor=black]{hyperref}



\usepackage{fancyhdr}
\lhead{\nouppercase{\leftmark}} \chead{} \rhead{\thepage}
\fancyfoot {}

\makeindex
%Glossar
\usepackage{nomencl}
\let\abbrev\nomenclature
\renewcommand{\nomname}{Glossar}
\setlength{\nomlabelwidth}{.25\hsize}
\renewcommand{\nomlabel}[1]{#1 \dotfill}
\setlength{\nomitemsep}{-\parsep}
\makeglossary


\usepackage[normalem]{ulem}
\newcommand{\markup}[1]{\uline{#1}}

\begin{document}
	
	% Trennungen
	\hyphenation{Wort-um-bruch}
\hyphenation{Kom-po-nen-ten-mo-dell}
\hyphenation{Trans-for-ma-tions-an-wei-sun-gen}
\hyphenation{Trans-for-ma-tions-an-wei-sung}
% \hyphenation{Un-ter-st"ut-zung}
\hyphenation{Mo-dell}
\hyphenation{Con-straints}
\hyphenation{Grund-an-nah-men}
\hyphenation{Grund-an-nah-me}
\hyphenation{Pro-jekt-ri-si-ken}
% \hyphenation{be-n\"o-tig-ten}
	
	% Titelblatt
	% $Log$
% Revision 1.2  2006/01/03 16:45:21  kelsaka
% - korrekturen eingepflegt: v. a. n�here definition der iterationsschritte
%
% Revision 1.1  2006/01/03 10:37:55  kelsaka
% - move to folder proposal
%
% Revision 1.1  2005/12/30 15:16:19  kelsaka
% initial creation
%

\begin{titlepage}


\begin{tabular} {p{0.50\textwidth} p{0.50\textwidth}}
\begin{flushleft}
	\includegraphics[width=0.45\textwidth]{image/logo-uni-ol.pdf}
\end{flushleft}
 &
\begin{center}
	\includegraphics[width=0.15\textwidth]{image/Palladio-Logo-03.pdf}
\end{center}
\end{tabular}

\begin{center}

\vfill {\ \\ \ \\ \ \\}

\vfill {\large{Proposal einer Diplomarbeit}}

\vfill {\large \normalfont --------------------------------------------------------------------------------------------\\
\textsc{\LARGE Entwicklung und Transformation \\ \tiny $\,$ \\ \LARGE eines EMF-Modells des \\ \tiny $\,$ \\ \LARGE Palladio Komponenten-Meta-Modells \ \\
\large \normalfont --------------------------------------------------------------------------------------------}}

\vfill {\ \\ \ \\ \ \\ \ \\ \ \\ \ \\ \ \\ \ \\ \ \\ \ \\}


\vfill{
\begin{flushleft}
\textbf{Bearbeitet von:} \\
Klaus Krogmann \\
Reinekeweg 2 \\
26676 Harkebr�gge \\
\texttt{kelsaka@gmx.de}
\end{flushleft}
}

\vfill {
\begin{flushleft}
\textbf{Betreut von:} \\
Erstgutachter: Jun.-Prof. Dr. Ralf Reussner \\
Zweitgutachter: Prof. Dr. Wilhelm Hasselbring \\
Betreuer: Dipl.-Wirtsch.-Inform. Steffen Becker \\
Fk. II, Department f�r Informatik, Abt. Software Engineering\\
an der Carl-Von-Ossietzky Universit�t, Oldenburg
\end{flushleft}
}

\vfill 
\begin{flushleft}
	\begin{tabular}{l}
				$$Revision$$\\
				$$Date$$ \\
				Status: Release
	\end{tabular}
\end{flushleft}

\end{center}
\end{titlepage}

	% Pagecounter auf 2 Setzen
	\setcounter{page}{2}
	

  %	leere Seite nach der Titelseite
	$\,$
	\newpage
		
	% Inhaltsverzeichnis
	\tableofcontents
	
	%	leere Seite nach TOC
	%\pagebreak
	%\newpage
	%$\,$

	\newpage
	% Fancyheader setzen
	\pagestyle{fancy}
		
	%Inhalt
	% $Log$
% Revision 1.1  2006/01/03 10:37:55  kelsaka
% - move to folder proposal
%
% Revision 1.1  2005/12/30 15:16:19  kelsaka
% initial creation
%
%\section{Abstract}



\section{Einleitung}
\label{sec:Einleitung}
Das vorliegende Proposal beschreibt die Planungen zur Umsetzung der Diplomarbeit "`Entwicklung und Transformation eines EMF-Modells des Palladio Komponenten-Meta-Modells"'. In der \textit{Einleitung} werden grundlegende Konzepte, Begriffe und verwendete Werkzeuge grob erkl�rt. Im anschlie�enden Teil \textit{Konzeption} werden der geplante Prozess zur Umsetzung skizziert, die grundlegenden Fragestellungen der Diplomarbeit diskutiert, Annahmen abgekl�rt und m�gliche Risiken aufgezeigt. Im letzten Teil des Dokuments \textit{Durchf�hrung} werden Formalia der Arbeit gekl�rt und eine Zeitplanung vorgestellt.

Ist im Folgenden von \textit{Komponentenmodell} die Rede, so ist, sofern nicht anders angegeben, das \textit{Palladio Komponentenmodell} gemeint. Als Palladio Komponentenmodell wird dabei das \textit{Meta}-Modell zur Darstellung von Komponentenarchitekturen der Palladio-Gruppe \cite{PALL} bezeichnet. Dieses Modell wird in Kapitel \ref{sec:DasPalladioKomponentenmodell} n�her beschrieben.

Zudem bezeichnet der Begriff \textit{Projekt}, solange er ohne sonstigen Kontext verwendet wird, die Diplomarbeit.



\subsection{Das Palladio Komponentenmodell}
\label{sec:DasPalladioKomponentenmodell}
Im Folgenden soll eine kurze Einf�hrung in die Konzepte des Palladio Komponentenmodells gegeben werden.

Das Palladio Komponentenmodell -- unter \cite{BECK} findet sich die Beschreibung einer �lteren Fassung des Modells -- beschreibt Software-Architekturen als eine Menge von Komponenten und Schnittstellen, sowie darauf definierten Relationen. Komponenten werden dabei in \textit{Basic Components} und \textit{Composite Components} unterschieden. Wie die Namen bereits andeuten, unterst�tzt das Komponentenmodell zusammengesetzte Strukturen. Das bedeutet, dass Komponenten und Schnittstellen mit beliebiger Tiefe rekursiv in \textit{Composite Components} geschachtelt werden k�nnen. In diesem Falle l��t sich die "`enth�lt"' Relation als Realisierung einer Komponente interpretieren.

Komponenten k�nnen Schnittstellen anbieten (\textit{Provides Interfaces}) oder ben�tigen (\textit{Requires Interfaces}).

Daneben wird die Komponenten- und Schnittstellen-Struktur in verschiedene Ebenen unterteilt, wor�ber sich Typ"=Konformit�ts"=Anforderungen definieren lassen. Auf der obersten Ebene wird vom \textit{Provided Type} verlangt, zu deklarieren, welche Schnittstellen angeboten werden. Darunter liegende Ebenen verfeinern im Allgemeinen unter Erhaltung der Typ"=Konformit�t den Typ einer h�heren Ebene. �ber diesen Mechanismus lassen sich Software-Architekturen schrittweise verfeinern und mit mehr Informationen anreichern.

Zus�tzlich zu den beschriebenen statischen Strukturen bietet das Komponentenmodell Unterst�tzung f�r beliebige Zusatzattribute, die f�r Entit�ten (Schnittstellen und Komponenten, Protokolle u. a.) vergegeben werden k�nnen. Hierunter fallen ebenfalls QoS (\textit{Quality of Service}) Attribute, f�r die das Komponentenmodell eine Berechnungsgrundlage bietet. Zu den F�higkeiten des Komponentenmodells z�hlen parametrisierte Vertr�ge, wie sie von Reussner in \cite{reussner01i} beschrieben werden, die Angabe von SEFFs (\textit{Service Effect Spezifikationen}, Spezifikation der externen Effekte von Dienstaufrufen auf einer Komponente) und die Verarbeitung von Protokollen auf Diensten einer Schnittstelle.

Ein neues Paper, das das aktuelle Komponentenmodell beschreibt, wird in K�rze unter \cite{PalladioCMNeu} ver�ffentlicht.



\subsection{Eclipse}
\label{sec:Eclipse}
Eclipse, wie auch alle anderen zu verwendenden Werkzeuge (zumeist Eclipse"=Plugins) im Bereich der Meta-Modellierung und anschlie�enden Modell-Transformation, befinden sich in einer stetigen Entwicklung. Der Charakter von Open-Source-Projekten macht die Entwicklungsarbeit in Teilen vom Engagement freiwilliger Entwickler ab\-h�n\-gig. Dies hat zur Folge, dass die Wachstumsgeschwindigkeit dieser Projekte stark variabel sein kann, die Erweiterungen zugleich aber auch neue M�glichkeiten bieten k�nnen, die zum Zeitpunkt der Erstellung dieses Propsals nicht absehbar sind.

Ebenso sind m�glicherweise nicht alle verf�gbaren Produkte bekannt, die sich zur Umsetzung der Diplomarbeit eignen w�rden. Somit kann die Auswahl geeigneter Werkzeuge lediglich eine unvollst�ndige  Momentaufnahme in der Marktbeobachtung darstellen. �nderungen in der Verf�gbarkeit von Werkzeugen haben somit direkt Einflu� auf die Diplomarbeit, sind zugleich aber nicht vorhersehbar.

Eine ausf�hrliche Absch�tzung zu den allgemeinen Risiken findet sich in Kapitel \ref{sec:Risikomanagement}.



\subsubsection{Entwicklung des Eclipse-Projekts}
\label{sec:EntwicklungdesEclipse-Projekts}
Eclipse \cite{eclipse} stellt ein umfangreiches Open Source Framework unter Java dar. Als solches stellt Eclipse diverse Plugin-Schnittstellen zur Verf�gung, f�r die bereits eine Vielzahl Plugins existieren.

W�hrend das prim�re Ziel von Eclipse die Bereitstellung einer Entwicklungsumgebung ist, wurden daneben mehrere Unterprojekte ausgegliedert, die sich des Eclipse"=Frameworks bedienen und f�r MDA-Prozesse (\textit{Model Driven Architecture}) Un\-ter\-st�t\-zung bereit stellen.

Das Eclipse-Projekt wird ma�geblich durch IBM \cite{IBM} gef�rdert. So basieren auch neuere kommerzielle Entwicklungen wie der IBM Rational Software Architect 6 oder Borland Together 2006 auf dem Eclipse-Framework. Dies verdeutlicht die Bedeutung Eclipses f�r industrielle Anwendungen, f�hrt andererseits aber auch dazu, dass sich die genannten Firmen, neben weiteren, finanziell und personell f�r das Eclipse-Projekt engagieren. Somit ist zugleich eine Weiterentwicklung des Projekts m�glich, die Ideen und Anregungen aus der Open Source Community aufgreift und mit den Mitteln der Industrie umsetzt.

Startete das Eclipse-Projekt im November 2001 noch mit acht Gr�ndungsmitgliedern, waren es Ende 2003 bereits 80 und heute bereits �ber 115 Mitglieder (vgl. \cite{eclipse-history}). Das anhaltende Wachstum und die Diversifikation �ber mehrere Sub-Projekte erscheint somit auch f�r die Zukunft gesichert.



\subsubsection{Eclipse Sub-Projekte}
\label{sec:EclipseSub-Projekte}
Im Rahmen der Diplomarbeit werden unter anderem die folgenden Sub-Projekte der Eclipse Foundation verwendet: EMF - Eclipse Modeling Framework \cite{EMF}, GEF - Graphical Editing Framework \cite{GEF}, GMF - Graphical Modeling Framework \cite{GMF} und der Merlin Generator \cite{Merlin}.



\subsubsection{Verwendung im Palladio-Kontext}
\label{sec:VerwendungimPalladio-Kontext}
Die DFG Nachwuchsgruppe Palladio \cite{PALL}, in deren Rahmen diese Diplomarbeit entsteht, entwickelt seit dem Jahr 2003 unter anderem ein Komponentenmodell zur ingenieursm��igen Evaluation von Software"=Architekturen, zur Vorhersage nicht"=funktionaler Eigenschaften und Untersuchung von Modelltransformationen (siehe auch Kapitel \ref{sec:DasPalladioKomponentenmodell}). Zur Validierung wissenschaftlicher Erkenntnisse wurde neben der konzeptionellen Entwicklung stets eine Referenz"=Implementierung des Palladio Komponentenmodells gepflegt. Erschien die Entwicklung in der Vergangenheit unter Microsofts .NET sinnvoll, vor allem weil die verwendete Programmiersprache C\# 1.0 gegen�ber Java 1.4 ben�tigte Spracherweiterungen bot (bspw. Klassen"=Attribute), so zeigt bereits die in Kapitel  \ref{sec:EclipseSub-Projekte}, aufgef�hrte Palette existierender Eclipse-Entwicklungen, dass f�r den ebenfalls forschungsrelevanten Bereich der Modelltransformationen eine weitaus breitere Unterst�tzung unter Eclipse existiert. Zudem ist das in Java 1.4 als einschr�nkend empfundene Fehlen von Klassen"=Attributen unter Java 1.5 aufgehoben, da Java mittlerweile �hnliche Konzepte bietet.

So sprechen folgende Argumente f�r die Verwendung von Eclipse:

\begin{itemize}
	\item Das Basis-Framework Eclipse stellt die M�glichkeit bereit, eigene Plugins zu entwickeln um ben�tigten Funktionsumfang zu erg�nzen. Daneben kann Eclipse als Basis f�r eigene selbst�ndige Applikationen dienen (Rich Client Platform \cite{rcp}).
	\item Insbesondere EMF (eine detailliertere Erkl�rung zu den hier aufgef�hrten Eclipse"=Projekten findet sich in Kapitel \ref{sec:EMF}) scheint als zentrales Meta-Modell zur Beschreibung des Palladio Metamodells (des Komponentenmodells) geeignet. Zudem bietet EMF die M�glichkeit generativ Modell-Quellcode, Queries und weiteres (siehe Kapitel \ref{sec:Gesamtprozess}) zu erzeugen.
	\item Daran anschlie�end erleichtern Modelltransformatoren wie Merlin und GMF die Erzeugung eines graphischen GEF-Editors.
	\item Schlie�lich stellt Eclipse mit GEF ein umfangreiches GUI-Framework bereit. Die Verwendung von GEF in kommerziellen Produkten wie Borland Together Software Architect 2006 und IBM Rational Architect 6 zeigt die praktische Relevanz von GEF.
\end{itemize}

Die .NET-Umgebung liegt in all diesen Bereichen deutlich zur�ck. Es gibt keine freien Entsprechungen zu EMF, Modelltransformationstools oder freien Grafikbibliotheken mit der M�chtigkeit der Eclipse Sub"=Projekte. Der Wechsel von .NET zu Java und Eclipse erscheint unter diesen Gesichtspunkten sinnvoll:

\begin{itemize}
	\item Die M�glichkeit auf die Generierung von Java-Quellcode zu einem Meta-Modell zur�ckzugreifen verspricht einen deutlichen Zeitgewinn, insbesondere bei einem h�ufigen Wechsel des zu Grunde liegenden Meta-Modells. Die Erfahrung innerhalb der Palladio"=Gruppe verdeutlicht, dass �nderungen des Komponentenmodells h�ufig vorgenommen wurden und werden. Dies ist wenig verwunderlich, zumal das Komponentenmodell zentraler Gegenstand der Forschung der Palladio"=Gruppe ist.
	\item GEF stellt ein umfangreiches Framework zur graphischen Anzeige und Bearbeitung von Graphiken bereit. Das Framework ist erweiterbar und verf�gt �ber umfangreiche Layout- und Rendering"=Werkzeuge. Die Architektur folgt dabei dem Model"=View"=Controller"=Prinzip. GEF ist dom�nenneutral und daher f�r eine breite Palette von Applikationen geeignet. Standardm��ig werden bereits Aktivit�tsdiagramme, GUI"=Builder, Klassendiagramm"=Editoren, Zustandsmaschinen und WYSIWYG Text-Editoren unterst�tzt.
	\item GMF und Merlin verfolgen einen generativen Ansatz, um zu bestehenden EMF-Meta-Modellen GEF-Editoren zu erzeugen. Merlin beschr�nkt sich dabei nicht nur auf das Generieren von GEF-Editoren. Dagegen versteht sich GMF als "`generative Br�cke zwischen EMF und GEF"' und zielt speziell auf die Erzeugung m�chtiger, flexibel generierbarer Editoren. Siehe hierzu auch Kapitel \ref{sec:Merlin-GMF}.
\end{itemize}



\subsection{MOF und EMF}
\label{sec:MOF}
\label{sec:EMF}
MOF, die Meta Object Facility \cite{MOF-Spezifikation} der OMG, bietet eine Standard-Schnittstelle f�r Meta-Modellierung an. Entsprechend der Philosophie von MDA (\textit{Model Driven Architecture}, \cite{MDA}), ist MOF auf Plattformunabh�ngigkeit und Wiederverwendbarkeit bedacht. Der als Framework ausgelegte Standard beinhaltet Mechanismen zum Erzeugen, Manipulieren, Finden, �ndern und Zerst�ren von Objekten und Beziehungen zwischen diesen Objekten. Die Menge g�ltiger Aktionen wird dabei �ber Meta-Modelle festgelegt.

In der aktuellen Spezifikation MOF 2.0 wird in Essential MOF (EMOF) und Complete MOF (CMOF) unterschieden. EMOF zielt dabei prim�r auf eine Erleichterung der Unterst�tzung durch Tools. Dazu wurde EMOF im Funktionsumfang reduziert und beschr�nkt sich auf die Kernkonzepte von MOF (\textit{Capabilities}): \textit{Reflection}, \textit{Identifiers}, \textit{Extension} und \textit{PrimitiveTypes}. Hinter \textit{Reflection} verbirgt sich ein Mechanismus zum Erkennen und Manipulieren von Meta-Objekten und Meta-Daten ohne initiales Wissen �ber Objekte, �hnlich dem Reflection-Mechanismus aus Microsofts C\#. \textit{Identifier} sind derzeit als URIs (Uniform Resource Identifier) realisiert und garantieren eine eindeutige Identifikation von Objekten, die auch �ber Transforms und Serialisierungen hinweg m�glich ist. \textit{Extensions} stellen eine M�glichkeit bereit, Objekte dynamisch zu anotieren und unvorhersehbare Informationen �ber Key-Value-Paare zu erg�nzen. \textit{PrimitiveTypes} stellen schlie�lich einen Basissatz von primitiven Datentypen (etwa String, Integer) zur Verf�gung.

EMF, das Eclipse Modeling Framework \cite{EMF}, ist in der aktuellen Version eine Java-Implementierung des EMOF-Standards, mit nur kleineren Abweichungen insbesondere bei der Benennung von Konzepten. EMF steht als Eclipse Plugin zum freien Download bereit. Das Basis-Datenmodell von EMF ist ECORE. Die N�he zwischen EMOF und EMF geht so weit, dass EMF EMOF transparent lesen und schreiben kann.

F�r die Diplomarbeit bietet EMF den Vorteil, dass es sich nahtlos in Eclipse integriert, zugleich aber die Implementierung des weit verbreiteten EMOF-Standards ist. Damit bleibt die Kompatiblit�t zu anderen EMOF-Implementierungen erhalten, was sp�ter einen einfachen Datenaustausch erm�glichen kann.



\subsection{Modellierung und Transformation}
\label{sec:ModellierungundTransformation}



\subsubsection{Transformation}
\label{sec:Transformation}
Transformationen werden in drei Klassen unterschieden:
\begin{itemize}
	\item \textbf{Model-to-Model} Werden zwei Modelle in einander �berf�hrt, so spricht man von einer Model-to-Model Transformation. Der Import von ECORE-Modellen in EMF entspricht einem solchen Transformationsvorgang.
	\item \textbf{Model-to-Text} Werden Modelle auf eine textuelle ("`flache"') Repr�sentation transformiert, so spricht man von Model-to-Text. Die Verwendung von JET (\textit{Java Emitter Templates}) in Merlin erm�glicht genau diese Form von Transformation. Mit Hilfe von JET-Templates wird zu einem gegebenen Modell eine GEF-Editor erzeugt. Auch GMF bietet eine Form von Model-to-Text Transformation, indem ebenfalls zu einem gegebenem Modell ein GEF-Editor generiert wird.
	\item \textbf{Text-to-Model} Unter Text-to-Model kann der Importvorgang von Quellcode mit anschlie�ender �berf�hrung in ein Modell verstanden werden. Im Sinne eines Roundtrip-Zyklus zwischen Quellcode und Modell werden Modellinformationen aus Quellcode extrahiert.
\end{itemize}
Die Transformation von Text zu Text ist im Rahmen der Diplomarbeit nicht von Bedeutung. Anhand des Prozesses, der in Kapitel \ref{sec:Gesamtprozess} dargestellt wird, lassen sich die unterschiedlichen Transformationsvorg�nge leicht nachvollziehen.



\subsubsection{Merlin / GMF}
\label{sec:Merlin-GMF}
Der Merlin-Generator \cite{Merlin} ist ein freies Eclipse Plugin, das auf EMF basiert. Dem MDA-Ansatz folgend, erm�glicht Merlin die Definition erweiterter Mappings f�r Java Templates (JET) und Model-to-Model Transformationen, wie zum Beispiel in ECORE, von wo aus eine Quellcode-Generierung angesto�en werden kann.

GMF, das Graphical Modeling Framework \cite{GMF}, versucht, wie auch Merlin, die Br�cke zwischen EMF und GEF zu schlagen. Hinter GMF stehen Branchengr��en wie IBM und Borland, die das Projekt mit professionellen Mitarbeitern f�rdern. Ziel des Projekts ist die Schaffung einer gemeinsamen Infrastruktur f�r EMF-basierte Modelle zur Erzeugung GEF-basierter GUIs.



\section{Konzeption}
\label{sec:Konzeption}
Die Idee hinter der Diplomarbeit ist die M�glichkeit, einen GUI-Editor nur auf Basis eines Meta-Modells sowie detaillierter Transformationsanweisungen erzeugen zu k�nnen. �nderungen am Meta-Modell k�nnten sich somit, solange sich die �nderungen in einem gewissen Rahmen bewegen, rein �ber einen neuerlichen Generierungsprozess in einem GUI-Editor wiederfinden. Damit w�rden Modifikationen schnell und einfach m�glich.

Damit die skizzierte Idee sinnvoll durchgef�hrt werden kann, darf jedoch der Aufwand (zeitlich und personell) f�r den Durchlauf des generativen Ansatzes nicht den Aufwand einer manuellen Implementierung �berschreiten. Der Break-Even-Point ergibt sich demnach aus den folgenden Faktoren:

\begin{itemize}
	\item Generativer Ansatz	
	\begin{enumerate}
		\item Initiale Erstellung des Meta"=Modells
		\item Pflege des Meta-Modells
		\item Initiale Definition von Transformationen
		\item Anpassung der Transformationen an die eigenen Bed�rfnisse (\textit{customizing})
		\item Pflege der Transformationsanweisungen
		\item H�ufigkeit der �nderungen des Meta-Modells
		\item "`Technologietreue"'; wie h�ufig werden die verwendeten Techniken gewechselt?
		\item "`Intensit�t"' der �nderungen; liegt ein Paradigmenwechsel vor, der die Anpassung von Transformationen verhindert?
		\item Aufwand f�r die Durchf�hrung eines Generierungsdurchlaufs
		\item Aufwand f�r die "`manuelle Nachpflege"' eines Generierungsdurchlaufs
	\end{enumerate}
	\item Manuelle Implementierung
	\begin{enumerate}
		\item H�ufigkeit der �nderungen des Meta-Modells
		\item Aufwand der Code-Pflege nach einer Modifikation des Meta"=Modells
		\item "`Technologietreue"'; wie h�ufig werden die verwendeten Techniken gewechselt?
		\item Aufwand f�r die Anpassung existierender Frameworks, die verwendet werden
	\end{enumerate}
\end{itemize}

Ein manuell erzeugter Editor d�rfte dabei zumeist den Vorteil haben, flexibler anpassbar zu sein und den eigenen Visionen und Bed�rfnissen einer Editor-Architektur folgen zu k�nnen. Spezielle eigene Funktionen k�nnen einem manuell erzeugten Editor einfacher hinzugef�gt werden, da dazu keine Templates angepasst werden m�ssten. Der Aufwand bei einem Technologie"=Wechsel ist wahrscheinlich kleiner als im Vergleichsfall des generativen Ansatzes, da umfangreichere �nderungen in der Architektur ohnehin ber�cksichtigt werden m�ssten.

Den gr��ten Einflu� auf die Entscheidung, den manuellen Ansatz oder den generativen Ansatz zu w�hlen, hat die H�ufigkeit der �nderungen am Meta-Modell und damit verbunden der Aufwand, einen generierten Editor manuell nachzupflegen.

Im Rahmen eingehender �berlegungen in der Palladio"=Gruppe im Vorfeld der Diplomarbeit und extrapoliert aus den Erfahren mit dem Komponentenmodell in der Vergangenheit, scheint ein generativer Ansatz eine Aufwandsverminderung zu bedeuten. Eine Evaluation erfolgt somit en passant mit der Diplomarbeit.



\subsection{Geplanter Prozess}
\label{sec:GeplanterProzess}
In den folgenden beiden Unterkapiteln wird der f�r die Diplomarbeit geplante Prozess beschrieben. In Kapitel \ref{sec:Gesamtprozess} wird der Gesamtprozess beschrieben, der vom Ablauf der Modellierung des Komponenten-Meta-Modells �ber Transformationen bis hin zu einem lauff�higen GEF-Editor geht. Den Kern der Diplomarbeit bildet hingegen der in Kapitel \ref{sec:Modellierungsprozess} beschriebene Prozess: die Modellierung des Meta-Modells des Palladio Komponentenmodells in EMF.



\subsubsection{Gesamtprozess}
\label{sec:Gesamtprozess}


\begin{figure}[htbp]
	\centering
		\includegraphics[width=1.00\textwidth]{./image/prozess01.pdf}
	\caption{Skizze des Gesamt-Entwicklungsprozesses der Diplomarbeit zur Generierung eines GEF-Editors}
	\label{fig:prozess01}
\end{figure}

In Abbildung \ref{fig:prozess01} wird der angestrebte Entwicklungsprozess f�r die Diplomarbeit grob skizziert. Im Kern wird sich die Diplomarbeit dabei mit den im oberen Bereich der Abbildung dargestellten Teilen des Entwicklungsprozesses besch�ftigen, der Erzeugung eines Meta-Modells des Palladio Komponentenmodells.

Dabei l��t sich der Entwicklungsprozess in die folgenden Schritte gliedern (wobei Tools in blau, respektive dunkel und Artefakte in gelb, respektive hell in Abbildung \ref{fig:prozess01} dargestellt sind):

\begin{enumerate}
	\item F�r die initiale Beschreibung des Meta-Modells in UML 2 bieten sich verschiedene Alternativen an. Zum einen ist eine Modellierung mit Hilfe von IBM Rational Software Architect 6 denkbar, daneben ist eine Modellierung unter Borland Together Architect 2006 m�glich. In beiden F�llen ist ein Export zu UML2 und ECORE m�glich. Da beide Tools einen �hnlichen Funktionsumfang aufweisen und gleicherma�en in Eclipse integriert sind, d�rfte die Art der Verf�gbaren Lizenzen �ber die Verwendung entscheiden. Derzeit existiert f�r IBM Rational eine Lizensierung f�r das OFFIS \cite{OFFIS}, die im Rahmen der Arbeit verwendet werden kann.
	\item\label{enum:ECORE} Im zweiten Schritt liegt das Meta-Modell, wie bereits beschrieben, alternativ als ECORE- oder UML2-Modell vor. Parallel dazu besteht die Option anotiertes Java f�r den folgenden Import in EMF zu verwenden. An dieser Stelle erscheint derzeit die Verwendung von ECORE den anderen zu bevorzugen, da ECORE das Basis-Meta-Modell f�r EMF darstellt.
	Eine Nutzung von anotiertem Java ist initial ohnehin nicht m�glich, da dies zuvor eine einmalige Generierung von Java-Code voraussetzt (siehe hierzu auch Schritt \ref{enum:RoundtripZyklus}).
	\item Von zentraler Bedeutung f�r die Konzeption des Modellierungs-Prozesses ist die Verwendung von Eclipse, erweitert um EMF als Modellframework. Als Eingaben bietet sich neben den bereits genannten M�glichkeiten in Form von ECORE, UML2 und anotiertem Java auch das Serialisierungsformat von IBM Rational an. Aus Gr�nden der Kompatiblit�t und der Offenheit f�r den Datenaustausch mit anderen Modellierungstools wird das Rational Datenformat jedoch nicht ber�cksichtigt, da es sich um ein propriet�res Datenformat handelt.
	\item Das Kernmodell f�r das zu entwickelnde Meta-Modell soll eine EMF-Repr�sentation darstellen. In der Abbildung \ref{fig:prozess01} wird dieses Artefakt daher auch mit fetten Lettern hervorgehoben. Insbesondere durch die M�glichkeit EMF im Rahmen eines Roundtrip-Prozesses einzubinden, wie er in den folgenden Schritten skizziert wird, zeigt sich die Eignung als Kerndatenmodell, das, wie genannt, �ber viele Wege durch Imports aufgebaut und durch die N�he zum EMOF-Standard weiterverwendet werden kann.
	\item Der mit \textit{EMF"=Generator} bezeichnete Kasten des Prozesses stellt einen wichtigen Aspekt f�r die M�glichkeit dar, ein Meta-Modell-Framework zu einem gegebenem EMF"=Modell zu generieren. Der "`EMF"=Generator"' setzt sich intern aus den Paketen \textit{core Framework} und \textit{EMF.Edit} zusammen.
	\item Insgesamt wird �ber den EMF-Generator ein vollwertiges Eclipse"=Plugin erzeugt, dass das vorgegebene Meta"=Modell bearbeitbar macht. Mit Hilfe des EMF"=Generators lassen sich die folgenden Teile eines Code"=Frameworks zu einem gegebenen Meta"=Modell generieren:
	
	\begin{itemize}
		\item \textbf{Klassen} Zu allen Elementen eines Meta-Modells kann Java-Quellcode erzeugt werden. Dabei werden alle in den UML-Diagrammen angegebenen Attribute in Getter / Setter �bersetzt und Methoden angelegt. Der Generator erzeugt zu allen Meta-Modell-Entit�ten eigene Interfaces sowie eine dazugeh�rige Implementierung. Auch manuelle �nderungen werden von diesem Generator ber�cksichtigt. Dazu wird f�r generierten Quellcode das \textit{Tag} \texttt{@generated} eingef�hrt, dass eben diesen generierten Code kennzeichnet. Wird das \textit{Tag} weggelassen, so bleiben die entsprechenden Code-Teile (bspw. nach manuellen �nderungen) erhalten, werden also nicht neu generiert.
		
		Die generierten Klassen erm�glichen automatisch eine konsistente Verwaltung von externen Referenzen. Hat also beispielsweise ein Auto eine Referenz auf seine Reifen, so k�nnen die Reifen gleicherma�en eine Referenz auf das Auto haben. Werden �nderungen an den Referenzen auf einer Seite vorgenommen, etwa das Entfernen der Zuordnung eines Reifens zu einem Auto, so sorgt ein Event-Mechanismus daf�r, dass sowohl Reifen als auch Auto aktualisiert werden.
		
		Im Vergleich zu der Implementierung des Palladio Komponentenmodells im November 2005 (C\#-Implementierung, siehe \cite{PalladioCMTechreportMH}) kann damit die Auftrennung in ein relationales Schema unter Benutzung von IDs (eindeutige Bezeichner) zur Abbildung der Entit�ten-Relationen und parallel dazu einer Sammlung aller vorhandenen Entit�ten entfallen. Vormals konnten die Entit�ten nur �ber einen speziellen Query-Mechanismus mit Hilfe ihrer ID erreicht werden.
		\item \textbf{Factories} Entsprechend dem Factory-Method Muster, wie es in \cite{GAMMA} beschrieben wird, wird zu jedem Modell �ber den EMF"=Generator ein Factory"=Interface nebst zugeh�riger Implementierung generiert. Die Instanz der Factory wird dabei als Singleton (siehe ebenfalls \cite{GAMMA}) gehalten, womit garantiert wird, dass zu keinem Zeitpunkt Instanzen des Modells unter Nutzung verschiedener Instanzen der gleichen Factory erzeugt werden k�nnen.
		
		Die Kapselung des Erzeugungsprozesses (Instanzierung der Modell-Klassen) wird zugleich gekapselt und dadurch einfacher austauschbar. Die Factory liefert lediglich Instanzen des Interfaces, jedoch nicht der internen Implementierungen zur�ck. Sollen andere Implementierung verwendet werden, muss lediglich die einmalige Factory-Instanzierung auf eine andere Implementierung gelenkt werden.
		\item \textbf{Model Wizard} Der \textit{Model Wizard} erm�glicht es, mit den in Eclipse �blichen Dialog-Wizards neue Modelle anzulegen. Damit ist die Bedienbarkeit konsistent zu den Eclipse-Standards.
		\item \textbf{Editor} Auf der GUI-Seite besteht der generierte Quellcode im Kern aus einem modell"=spezifischen Editor. Dieser integriert sich f�r die im Model Wizard generierten Modelle in Eclipse und kann dann zum Bearbeiten der Modelle verwendet werden. Der Editor besteht im Wesentlichen aus einer Baumansicht (\textit{Tree View}), in der kontextsensitiv Knoten eingeh�ngt werden k�nnen. Das heisst also, dass beispielsweise in einem Auto-Modell-Editor nur maximal vier Reifen zu einem Auto hinzuf�gbar sind.
		
		Neben der Baumansicht lassen sich zudem die Eigenschaften der Modell-Entit�ten im \textit{Propertysheet} von Eclipse bearbeiten. Dies sind im Allgemeinen die Modell"=Eigenschaften, die �ber Getter / Setter verf�gen.
		\item \textbf{Tests} Zu dem generierten Eclipse-Plugin werden automatisch Unit-Tests erzeugt, die im JUnit-Framework \cite{JUnit} ablauff�hig sind. Diese Tests umfassen das Erzeugen von Modell-Entit�ten sowie das Laden und Speichern eines Modell-Beispiels mit dem Editor-Plugin. Vor allem manuelle Anpassungen des generierten Codes werden dadurch fehlerrobuster m�glich. Durch ein permanentes Testen gegen die generierten Tests wird sichergestellt, dass wichtige Grundfunktionalit�ten stets erf�llt sind.
		\item \textbf{Commands} Damit zu einem Zeitpunkt mehrere Sichten auf ein Modell m�glich sind und �nderungen konsistent von allen Sichten (inklusive des "`Propertysheets"') erfasst werden k�nnen, wird automatisch eine Adpater-Schicht generiert, die eine command-�hnliche Funktion inne hat. F�r alle Setter wird bereits automatisch ein Event-Handling generiert, so dass �nderungen an Eigenschaften der Modell-Entit�ten von beliebigen Betrachtern beobachtet werden k�nnen. Um auch eine Unterst�tzung f�r das Eclipse-Framework au�erhalb des Modell-Editor-Plugins bieten zu k�nnen, werden die intern als \textit{EMF adapter notification} bezeichneten Aufrufe durch die Adapterschicht in \texttt{fireNotifyChanged} Ereignisse des Eclipse-Frameworks �bersetzt.
		
		Die Erzeugung von Adaptoren ist wiederum �ber Factories gekapselt.
		\item \textbf{Reflective API manipulation} EMF bietet eine Reflective API, um Instanzen eines Modells zu erzeugen (vgl. \cite{devx}), zu initialisieren und zuzugreifen oder dynamisch Modelle zu erstellen. Analog zur Java-Methode \texttt{Object.} \texttt{getClass()} bietet \texttt{eClass()} die M�glichkeit Meta-Daten einer Instanz (etwa die \texttt{EClass}) in Erfahrung zu bringen. Auf jedem Objekt kann der Zugriff auf Daten per \texttt{eGet()} und \texttt{eSet(}) erfolgen, �hnlich dem in Java �blichen Zugriff �ber \texttt{java.lang.reflect.Method.invoke()}. Zus�tzlich bietet jedes Instanz"=Objekt Zugriff auf seine Container (\textit{Parent}-Beziehung) �ber \texttt{eContainer()}. Da \texttt{EObject} von \texttt{Notifier} erbt, ist es m�glich alle �nderungen von Objekt-Daten zu �berwachen (\textit{monitoring}).
		\item \textbf{Validation"=Framework} EMF unterst�tzt die Validierung von Modell"=Constraints. Constraints k�nnen zum Beispiel �ber XML"=Schemata ausgedr�ckt und mit einem Bezeichner versehen werden. Im annotierten Java erfolgt dann eine Referenzierung der Constraints:
\begin{verbatim}
      @model
        annotation="[URI - http://www.here.com]
        constraints='[constraint name]'"
\end{verbatim}
im Bereich der JavaDocs.

Zudem ist die Definition von Constraints auch im generierten Java"=Quellcode m�glich. Siehe hierzu auch \cite{emf-validation}.
		\item \textbf{Queries} Queries erm�glichen das gezielte Durchsuchen von EMF-Modellen, um an Entit�ten des Modells zu gelangen oder Struktur-Informationen zu extrahieren.
	\end{itemize}
	
	\item\label{enum:RoundtripZyklus} Der bereits oben angedeutete Roundtrip-Zyklus (in der Abbildung \ref{fig:prozess01} als Schleife oben rechts erkennbar) schlie�t mit der Repr�sentation des Modells in anotiertem Java, das wiederum von EMF importiert werden kann. Damit ist eine Synchronisation zwischen EMF-Modell und der Modell-Repr�sentation im Java-Quellcode m�glich.
	
	Dennoch scheint dieser Roundtrip-Zyklus f�r die Diplomarbeit nicht angeraten, da die unter Schritt \ref{enum:ECORE} angef�hrte Modell-Repr�sentation im ECORE-Modell nicht konsistent mitgepflegt w�rde. Um auch diesen Schritt des Modellierungsprozesses konsistent zu halten, sollten Modelliterationen nicht �ber den Roundtrip-Zyklus vorgenommen werden, sondern stets mit Modifikationen des ECORE-Modells beginnen. Unter der Annahme, dass der generierte Quellcode nach einer Modell�nderungen problemlos mit manuellen �nderungen zusammengef�hrt (\textit{merge}) werden kann, l��t sich der skizzierte Prozess ohne den Re-Import von anotiertem Java stets wiederholen, wobei das Modell schrittweise verfeinert werden kann.
	
	Zusammengefasst erscheint eine Nutzung der Roundtrip-M�glichkeit mit dem Import von anotiertem Java zun�chst nicht sinnvoll, da dass ECORE-Modell nicht in den Roundtrip-Zyklus einbezogen werden k�nnte.
	
	\item Der n�chste Schritt des Prozesses ist die Transformation unter Verwendung von GMF bzw. Merlin. Wie bereits in Kapitel \ref{sec:Merlin-GMF} angesprochen, zielen Merlin und GMF zu Teilen in die gleiche Richtung: "`Eine generative Br�cke zwischen EMF und GEF"'.
	
	W�hrend Merlin in erster Linie Code-Generierung und Model-Transformationen unterst�tzt, zielt GMF direkt auf die Erzeugung von GEF-Editoren. Im Merlin-Kontext nimmt die GEF-Generierung derzeit eine Nebenrolle ein. Hier sind vordefinierte Templates verf�gbar, die einen einfachen graphischen Editor erzeugen.
	
	Wie bereits oben angesprochen, gibt es Bestrebungen, Merlin in das GMF-Projekt zu �berf�hren. Da GMF zum derzeitigen Zeitpunkt bereits m�chtiger als Merlin erscheint und laufend an Stabilit�t gewinnt, ist mit einer sinnvoll verwendbaren Version f�r die Zeit der geplanten Transformationen zu rechnen. Vor diesem Hintergrund bietet sich die Verwendung von GMF f�r die Transformationsprozesse in der Diplomarbeit an.
	
	Wie in Abbildung \ref{fig:prozess01} angedeutet, sind zur gezielten Erzeugung eines GEF"=Editors zus�tzliche Transformationsanweisungen (\textit{Transformations}) notwendig. In diesen Transformationsanweisungen muss unter anderem hinterlegt werden, ob und wie Assoziationsklassen  dargestellt werden sollen, welche Darstellungssymbole verwendet werden sollen und welche Elemente in der Werkzeugleiste erscheinen sollen. Die Transformationsanweisungen flie�en schlie�lich in den generierten GEF-Editor ein.
	
	\item Der GEF-Editor stellt das nahezu fertige Produkt dar. Realistischer Weise ist jedoch davon auszugehen, dass trotz stark anpassbarer Transformationen kleinere �nderungen manuell nachzupflegen bleiben. Idealerweise ist der hier notwendige Aufwand m�glichst klein. Da diese manuellen Modifikationen nach jeder �nderungen des ECORE-Modells durchzuf�hren w�ren, entscheidet sich an dieser Stelle, ob der generative Ansatz leistungsf�higer als der der vollst�ndig manuellen Implementierung ist.
	
	\item Der Editor ist das Endprodukt, dass jederzeit nach �nderungen am zu Grunde liegenden ECORE-Modell durch einen neuerlichen Durchlauf des Generierungsprozesses entstehen soll. Dieses Produkt muss sich unter anderem mit dem Ergebnis der Projektgruppe Ride.NET \cite{RideDotNet} messen lassen, deren Ziel die Implementierung eines Editors des Palladio Komponentenmodells unter C\# war.
\end{enumerate}



\subsubsection{Modellierungsprozess}
\label{sec:Modellierungsprozess}

\begin{figure}[htbp]
	\centering
		\includegraphics[width=1.00\textwidth]{./image/model-process01.pdf}
	\caption{Modellierungs--Prozess}
	\label{fig:model-process01}
\end{figure}

Im Kern soll sich die Diplomarbeit mit der Meta-Modellierung des Palladio Komponentenmodells besch�ftigen. In Abbildung \ref{fig:model-process01} wird eine iterative Variante zur Entwicklung des Meta-Modells dargestellt. Dazu wird der in Abbildung \ref{fig:prozess01} dargestellte obere Bereich hervorgehoben. Im Gegensatz zur Darstellung des Gesamtprozesses wird jedoch bewusst der Roundtrip-Zyklus �ber den Import von anotiertem Java ausgeblendet. Wie bereits im vorigen Kapitel erl�utert wurde, w�rde bei diesem Vorgehen die ECORE-Modell-Darstellung "`out-of-sync"' geraten.

Statt dessen ist eine Evaluation der generierten Modelle mittels kleinerer Modellierungstests vorgesehen. Durch mehrere Iterationen soll das Meta-Modell des Palladio Komponentenmodells verfeinert werden. Dieses Verfahren ist vor allem notwendig, weil es bisher keine abgeschlossene vollst�ndige formale Darstellung des Palladio Komponentenmodells gibt. Daher muss der skizzierte Modellierungs-Prozess zugleich als ein Verfahren zur Ermittlung von Anforderungen aufgefasst werden. In diesem Verfahren geht es darum, zu pr�fen, ob sich alle (theoretischen) Konzepte, die dem Palladio Komponentenmodell zu Grunde liegen, ebenfalls im EMF Meta"=Modell widerspiegeln, bzw. mit dem Meta"=Modell abbilden lassen.

Der iterative Prozess sieht daher wie folgt aus:
\begin{enumerate}
	\item Das ECORE-Modell wird initial modelliert.
	\item Aus dem ECORE-Modell wird mit Hilfe des EMF-Generators eine Abbildung auf Java-Code vorgenommen.
	\item In der Evaluationsphase wird gepr�ft, ob die Umsetzung des Modells
	\begin{itemize}
		\item vollst�ndig
		\item korrekt
		\item applizierbar
	\end{itemize}
	ist. Damit wird sichergestellt, ob sich alle Konzepte des Palladio Komponentenmodells umsetzen lassen. Oder es wird festgestellt, welche Konzepte sich nicht �ber ein ECORE-Modell darstellen lassen und welche Einschr�nkungen hierf�r der Grund sind. Zudem wird darauf geachtet, dass das richtige Meta-Modell entworfen wird und das Meta-Modell richtig entworfen wird. Nicht zuletzt f�hrt der praktische Test zu einem anwendbaren Modell, das sich f�r Test-Modelle bereits bew�hrt hat.
	
	Der iterative Ansatz ist der Wartbarkeit als zutr�glich aufzufassen. Da das Palladio Komponentenmodell in der Vergangenheit diverse �nderungen und Erweiterungen erfahren hat, ist davon auszugehen, dass auch zuk�nftig weitere Modifikationen zu erwarten sind. Durch die Wahl mehrerer Iterationsschritte wird die Wartbarkeit / �nderungsfreundlichkeit des zu entwickelnden Meta-Modells unmittelbar getestet.
\end{enumerate}

Zum jetzigen Zeitpunkt ist bereits absehbar, dass die Erweiterbarkeit des Meta-Modells um variable, zuvor nicht typisierte Zusatzattribute  Teil eines fest geplanten Iterationsschrittes sein wird. Siehe hierzu auch Kapitel \ref{sec:Fragestellungen}.

Das in der Abbildung \ref{fig:model-process01} angef�hrte "`Feedback"' wird von den Mitgliedern der Forschungsgruppe Palladio und vom Diplomanden durch eine kritische Reflexion vorgenommen. Eine klare konzeptionelle Abgrenzung des Komponentenmodells ist dabei ebenfalls Teil der Diplomarbeit.



\subsection{Fragestellungen}
\label{sec:Fragestellungen}
Der Diplomarbeit liegen unter anderem die folgenden Fragestellungen zu Grunde:
\begin{itemize}
	\item Widersprechen sich Modellierungskonzepte des Palladio Komponentenmodells? Sind Einschr�nkungen des Komponentenmodells erkennbar? Wurden auf Grund fehlender formaler Manifestierung des Modells Unvollst�ndigkeiten �bersehen? Welche Semantik verbirgt sich hinter spezifischen Modellierungskonstrukten?
	\item Das Palladio Komponentenmodell ist zu kleinen Teilen nur implizit in den K�pfen der Mitglieder der Palladio-Gruppe vorhanden. Gibt es Widerspr�che in den impliziten Annahmen?
	\item Welche der konzeptionellen Aspekte des Palladio Komponentenmodells lassen sich mit den gew�hlten Hilfsmitteln wie UML, ECORE, EMF und anotiertem Java abbilden?
	\item Welche Aspekte lassen sich nicht abbilden und aus welchen Gr�nden scheitert diese Abbildung. Welche Einschr�nkungen gelten f�r UML, ECORE, EMF und anotiertem Java? Welche Erweiterungen der (Abbildungs-) Modelle m�ssten vorgenommen werden.
	\item Welche Wege bieten sich zur Modellierung an? Welche Alternativen gibt es bei der Modellierung? Aus welchen Gr�nden wurden welche Alternativen gew�hlt? Gibt es gleichwertige Alternativen oder widerspr�chliche Herangehensweisen?
	\item Wie lassen sich variable Zusatzattribute im Meta-Modell unterbringen und wie finden sich diese Zusatzattribute im generierten Modell-Code bzw. einem Modell-Editor wieder?
	\item Welche technologischen und konzeptionellen Einschr�nkungen bringen die verwendeten Generatoren und Transformatoren mit sich?
\end{itemize}



\subsection{Annahmen}
\label{sec:Annahmen}
Um die Diplomarbeit entsprechend der Planungen aus Kapitel \ref{sec:GeplanterProzess} durchf�hren zu k�nnen, m�ssen einige Annahmen erf�llt sein. Diese Annahmen lassen sich nicht in vollem Umfang �berpr�fen, da dies wesentliche Teile der Diplomarbeit vorweg nehmen w�rde.

\begin{itemize}
	\item UML, ECORE, EMF und anotiertes Java m�ssen m�chtig genug sein, um darin alle Konzepte des Palladio Komponentenmodells abzubilden.
	\item Um die bereits angesprochenen variablen Zusatzattribute zu Modell-Entit�ten zu erm�glichen, ist es notwendig, dass dies durch die Meta-Modellierung auf den folgenden Ebenen unterst�tzt wird: UML, ECORE, EMF und anotiertem Java. F�r UML bedeutet dies, dass das Meta-Modell f�r ein gegebenes Zusatzattribut dynamisch vom Kern-Meta-Modell (Meta-Modell ohne Zusatzattribute) referenziert werden kann.
	\item Ferner wird angenommen, dass der zu verwendende Generator von EMF ausreichend m�chtig ist, um alle Aspekte des Palladio Komponentenmodells in anotiertem Java abzubilden.
	\item Die M�glichkeiten die Transformationen von GMF eigenen Bed�rfnissen anzupassen m�ssen ausreichend gro� sein, um einen den Benutzerbed�rfnissen entsprechenden Editor generieren zu k�nnen.
	\item Wie bereits angesprochen, muss davon ausgegangen werden, dass der in Kapitel \ref{sec:Gesamtprozess} skizzierte Prozess in mehreren Iterationen (insbesondere auf Grund der "`konzeptionellen Variabilit�t"' des Palladio Komponentenmodells) durchlaufen werden muss. Da nicht davon ausgegangen werden kann, dass jegliche generierten Modelle und Code ohne manuelle Nachpflege verwendet werden k�nnen, ergibt sich zwingend die Anforderung, dass der EMF-Generator und auch GMF funktionierende "`Merge"'-Mechanismen implementieren, die verschiedene Modellversionen (zumindest nach kleineren �nderungen) unter Erhaltung manueller �nderungen zusammenf�hren k�nnen. Nur mit einem solchen Mechanismus l�sst sich der iterative Prozess sinnvoll durchf�hren.
	
	Zum Zeitpunkt der Niederschrift dieses Proposals wird ein solcher Merge"=Mechanismus von Merlin nicht unterst�tzt.
\end{itemize}

Sind nicht alle hier dargestellten Annahmen erf�llt, hat dies direkte Auswirkungen auf die Umsetzung der Diplomarbeit. Daher sind die hier genannten Annahmen auch im Zusammenhang mit den Risiken zur Umsetzung der Diplomarbeit (siehe Kapitel \ref{sec:Risikomanagement}) zu sehen.



%\subsection{Fehlerquellen}
%\label{sec:Fehlerquellen}



\subsection{Einschr�nkungen}
\label{sec:Einschraenkungen}
Im Rahmen der Diplomarbeit ist \textit{nicht} geplant, einen vollst�ndigen, industriellen Standards gen�genden, GUI-Editor f�r das Palladio Komponentenmodell zu erzeugen. Im Vordergrund steht die Modellierung der EMF-Darstellung des dem Editor als Basis dienenden Komponenten-Meta-Modells und die Erprobung m�glicher Wege zur Generierung eines vollst�ndigen Editors.



\subsection{Risikomanagement}
\label{sec:Risikomanagement}
Die Durchf�hrung der Diplomarbeit unterliegt vielf�ltigen Risiken. �ndern sich Grundannahmen, wie in Kapitel \ref{sec:Annahmen} dargestellt, kann ein erfolgreicher Abschluss der Diplomarbeit gef�hrdet sein. Um Risiken zu minimieren werden an dieser Stelle f�r wahrscheinliche Risiken Auswege aufgezeigt.

\paragraph{Konzeptioneller Wandel im Palladio Komponentenmodell} Wie bereits mehrfach in vorangegangenen Kapiteln erw�hnt wurde, ist das Palladio Komponentenmodell aktiver Gegenstand von Forschungsarbeit. Damit verbunden ist das Risiko, dass sich Modell"=Konzepte �ndern k�nnen, Modell"=Erweiterungen erg�nzt werden und nicht mehr ben�tigte oder problematische Teile des Modells wegfallen k�nnen. Um das Risiko in diesem Bereich einzugrenzen, bieten sich zwei Strategien an:
\begin{enumerate}
	\item Wird die Version des Komponentenmodells f�r die Diplomarbeit eingefroren, kommen konzeptionelle �nderungen in der Diplomarbeit nicht zum Tragen. Dieser Ansatz birgt jedoch die Gefahr, dass die Modellversionen in der Diplomarbeit und der realen Entwicklung stark divergieren.
	\item �nderungen am Komponentenmodell k�nnen genau beobachtet werden und nur wenn �nderungen als unkritisch eingestuft werden, flie�en sie in die Diplomarbeit ein. Dies hat den Vorteil, dass das Meta-Modell aus der Diplomarbeit st�rker dem realen Stand entspricht.
\end{enumerate} 

\paragraph{Annahmen nicht erf�llt} Sind Annahmen, die in Kapitel \ref{sec:Annahmen} festgestellt wurden, nicht erf�llt, sind hiermit die gr��ten Projektrisiken verbunden. Je nachdem, zu welchem Zeitpunkt im Prozess die nicht erf�llten Annahmen zum Tragen kommen, ist anzunehmen, dass die Erf�llung aller sp�teren Teile des Prozesses nur noch unwahrscheinlich ist. Das bedeutet, dass Abstriche im Umfang der Diplomarbeit zu erwarten sind. Um diesen Ausfall abzumildern, k�nnten beispielsweise f�r die verwendeten Modelle und Werkzeuge Alternativ-Produkte verwendet werden.

\paragraph{Verf�gbarkeit verwendeter Werkzeuge} Sollten im Rahmen der Diplomarbeit verwendete Werkzeuge nicht mehr verf�gbar sein (etwa aus Gr�nden von Lizenzproblemen), sollte versucht werden, geeignete Alternativen zu finden und zu verwenden. Insbesondere im Falle von GMF baut die Diplomarbeit darauf, dass dieses Werkzeug, wie in der Vergangenheit, weiterentwickelt wird. Da hinter GMF ma�geblich IBM und Borland mit professionellen, bezahlten Entwicklern stehen, ist es notwendig, dass diese Firmen weiterhin Interesse an der Entwicklung haben und keinen strategischen Wechsel vollziehen. Zudem ist die Diplomarbeit davon abh�ngig, dass das Projekt ausreichend schnell voranschreitet und in einen stabilen Zustand gelangt.

Sollten Probleme mit GMF auftreten, bietet sich ein R�ckgriff auf Merlin oder die bis dahin letzte stabile Entwicklerversion an.

\paragraph{Vertikaler Prototyp}
Zur weiteren prophylaktischen Minimierung von Risiken wurde bereits erfolgreich (wenn auch mit gewissen Einschr�nkungen) ein vertikaler Prototyp unter Verwendung von Merlin erzeugt. Als einer der ersten Schritte der Diplomarbeit ist ein weiterer vertikaler Prototyp geplant, der sp�ter auch Basis darauf folgender Iterationen sein wird. Dieser soll GMF verwenden und variable Zusatzattribute testen.



\section{Durchf�hrung}
\label{sec:Durchfuehrung}
\subsection{Betreuer}
\label{sec:Betreuer}
\begin{itemize}
	\item \textbf{Erstgutachter:} Jun.-Prof. Dr. Ralf Reussner
	\item \textbf{Zweitgutachter:} Prof. Dr. Wilhelm Hasselbring
	\item \textbf{Betreuer:} Dipl.-Wirtsch.-Inform. Steffen Becker
\end{itemize}



\subsection{Artefakte}
\label{sec:Artefakte}
F�r den Verlauf der Diplomarbeit sind die folgenden Artefakte vorgesehen:
\begin{itemize}
	\item Proposal (dieses Dokument), \emph{\textbf{�}}
	\item Begleitfolien zum Proposal, \emph{\textbf{�}}
	\item Quellcode anl��lich der Iterationen laut Kapitel \ref{sec:Iterationsschritte}
	\item Diplomarbeit, \emph{\textbf{�}}
	\item Begleitfolien zur Diplomarbeit, \emph{\textbf{�}}
	\item Quelltexte und Modell-Serialisierungen der finalen Version des Meta-Modells, \emph{\textbf{�}}
\end{itemize}
\emph{\textbf{�}} kennzeichnet f�r die Ver�ffentlichung bestimmte Artefakte.


\subsection{Entwicklungsumgebung und Werkzeuge}
\label{sec:EntwicklungsumgebungundTools}
Als Entwicklungsumgebung wird Eclipse verwendet. Zus�tzlich zu verwendende Werkzeuge sind: Rational Software Architect sowie als Erg�nzung des Eclipse-Frameworks die Eclipse-Plugins EMF, GMF, Merlin, UML2, GEF und Omondo UML.

Daneben sind vorgesehen: LaTeX, TeXnicCenter, CVS und Visio 2003.



\subsection{Vorgehensmodell}
\label{sec:Vorgehensmodell}
Wie bereits im Kapitel \ref{sec:GeplanterProzess} genannt, wird f�r die Diplomarbeit ein iteratives und inkrementelles Vorgehen gew�hlt. Dies hat ebenfalls einen positiven Effekt auf die Projektrisiken, da bereits zu einem fr�hen Zeitpunkt ein lauff�higes Produkt zur Verf�gung steht, das ein g�nzliches Scheitern der Diplomarbeit verhindert.



\subsection{Zeitplanung}
\label{sec:Zeitplanung}
Entsprechend der Vorgaben der DPO4 vom 01.02.2002 betr�gt die Dauer der Diplomarbeit sechs Monate. Die Diplomarbeit beginnt am 01.01.2006 und endet entsprechend sp�testens am 31.06.2006. W�hrend der Diplomarbeit finden in der Regel w�chentliche Besprechungen mit den Betreuern statt.



\subsubsection{Gantt-Charts}
\label{sec:Gantt-Charts}
\begin{figure}[htbp]
	\centering
		\includegraphics[width=0.60\textwidth]{./image/zeitplanung01.pdf}
	\caption{Zeitplanung der Diplomarbeit}
	\label{fig:zeitplanung01}
\end{figure}
In Abbildung \ref{fig:zeitplanung01} wird der geplante zeitliche Verlauf der Diplomarbeit, inklusive aller wichtigen Meilensteine, dargestellt.

Die Termine f�r die Vortr�ge im DuD-Seminar stehen derzeit noch nicht fest, sind in der Gantt-Chart-Darstellung jedoch bereits enthalten.



\subsubsection{Iterationsschritte}
\label{sec:Iterationsschritte}
F�r die Diplomarbeit werden die folgenden Iterationsschritte angestrebt, deren zeitliche Terminierung Kapitel \ref{sec:Gantt-Charts} zu entnehmen ist:
\begin{itemize}
	\item Vertikaler Prototyp unter Verwendung von GMF und dynamischer Annotation
	\item 1. Iteration der Modellierung in EMF
	\item 2. Iteration der Modellierung in EMF
	\item GMF-Transformationen definieren, 1. Iteration; Editor-Funktionalit�t f�r das gesamte EMF-Modell herstellen
	\item GMF-Transformationen definieren, 2. Iteration; Anpassung des generierten Editors (\textit{Customizing})
	\item Manuelle Anpassung; eventuell vorhandene Fehler korrigieren
\end{itemize}






%____________________________________________________________________________________________________________
%____________________________________________________________________________________________________________
%____________________________________________________________________________________________________________
%
% "`Hineinlinken"' von Zusatzinformationen in UML muss m�glich sein. Zun�chst Linking und Referenzieren ausprobieren.
%
% Wichtig: Trennung von Shapes und Entit�ten
%
% KM3 Eclipse --> Ascii-Repr�sentation


	
	% Anhang
	% Abbildungsverzeichnis
	\addcontentsline{toc}{section}{Abbildungsverzeichnis}
	\lhead{Abbildungsverzeichnis} \chead{} \rhead{\thepage}
	\listoffigures
	

	%Literaturverzeichnis
	\newpage
	\addcontentsline{toc}{section}{Literaturverzeichnis}
	\lhead{Literaturverzeichnis} \chead{} \rhead{}
	% \cite* % Verursacht [] vor dem Literaturverzeichnis; ausserdem nur noch Anzeige verwendeter Literatur.
	\bibliographystyle{ieeetrans}
  \bibliography{Literaturverzeichnis}
  	
	% Glossar
	%\newpage
	%\lhead{Glossar} \chead{} \rhead{\thepage}
	%\addcontentsline{toc}{section}{Glossar}
	%\printglossary
	%% $Log$
% Revision 1.1  2006/01/05 16:46:14  kelsaka
% - initial creation
%
% Revision 1.1  2006/01/03 10:37:55  kelsaka
% - move to folder proposal
%
% Revision 1.1  2005/12/30 15:16:18  kelsaka
% initial creation
%
%


\abbrev{Tag}{Kennzeichnungssymbol TODO}

% Kann unsortiert bleiben!!!
%\abbrev{Administrator}{Verwalter eines Softwaresystems oder eines Softwareprogrammes.}
%\abbrev{Algorithmus}{Eine zur L�sung eines bestimmten Problems oder einer bestimmten Problemklasse m�gliche Vor\-ge\-hens\-wei\-se/-folge.}
\abbrev{Analyse}{Analysen werden in Ride.NET von Analyseplugins durchgef�hrt. Analysen ermitteln Zusatzinformationen zu einem gegebenen Modell wie beispielsweise Performance-Vorhersagen oder Kontrollfluss-Visualisierungen.}
\abbrev{Anforderungen}{Anspr�che, denen das zu entwickelnde Softwaresystem entsprechen muss. Die A. an
ein Softwaresystem werden in der Anforderungsdefinition festgehalten.}
\abbrev{Anforderungsdefinition}{Bei oder vor der Softwareentwicklung entstehendes Dokument, in dem
funktionale, nichtfunktionale und generelle Anforderungen an die zu entwickelnde Software festgehalten werden.
Die A. stellt oftmals die Vertragsgrundlage bei Entwicklungsauftr�gen von Softwaresystemen dar.}
\abbrev{Antwortzeit}{Im Zusammenhang mit der Benutzerschnittstelle stellt die Antwortzeit eine nichtfunktionale Anforderung dar, die angibt, nach welcher Zeit ein Benutzer die Antwort auf seine Eingaben vom System erh�lt. F�r umfangreichere Berechnungen w�re dies die Zeit, nach der die Berechnungen abgeschlossen sind und dem Benutzer angezeigt werden k�nnen. Siehe auch \textit{Reaktionszeit}.}
\abbrev{Anwendungsfall}{Beschreibung eines speziellen Benutzungsszenarios der zu entwickelnden Software.}
\abbrev{Anwendungsfalldiagramm}{Diagrammform in der UML, in der Anwendungsf�lle der zu entwickelnden Software dargestellt sind.}
\abbrev{Artefakte}{(auch Softwareartefakte) Umfasst die Dokumentation, den Quellcode und sonstige softwarebeschreibende Informationen. Zumeist ist die elektronische Form dieser Daten gemeint. Ausgedehnt auf ein Projekt umfassen die Artefakte alle Projektinformationen (siehe auch Projekt).}
%\abbrev{Benutzer}{auch Nutzer oder User; der (End-)Benutzer eines Softwaresystemes.}
\abbrev{Bibliotheken}{Sammlung von Programmen/Funktionen, die gebrauchsfertig bei Programmierung eines
Programms zur Verf�gung stehen (k�nnen), wenn sie im Lieferumfang eines Compilers enthalten sind. B. werden je
nach Bedarf durch einen Aufruf in
Programmsegmente eingebunden.}
%\abbrev{Browser}{z. B. Navigator von Netscape, Internet Explorer von Microsoft, Mosaic...  Zeigt die von einem Server bereitgestellten Dokumente (meist in HTML) an. Er dient auch als Plattform, um Applets auszuf�hren.}
\abbrev{Bug}{Fehler oder Absturz verursachender Quelltext in
einem Software-System.}
\abbrev{C\#}{Ausgesprochen: "`C Sharp"'. Moderne objektorientierte Programmiersprache aus der Feder von Microsoft mit starken Anlehnungen an Java und C++.}
\abbrev{C0-Test}{Bei diesem Test m�ssen alle Anweisungen eines Programms mindestens einmal durchlaufen werden (Anweisungs�berdeckung). Ziel des C0-Tests ist das Finden von Programmteilen, die nicht ausf�hrbar sind. Bei dem Test handelt es sich um ein notwendiges aber nicht hinreichendes Testkriterium. Vorzugsweise wird dieser Test beim Komponententest verwendet.}
\abbrev{C1-Test}{Der C1-Test verlangt, dass alle Ausg�nge aller Verzweigungselemente des zu testenden Objektes mindestens einmal durchlaufen werden m�ssen, d.h. jede Kante im Kontrollflussdiagramm des Testobjektes muss mindestens in einem Testfall einmal durchlaufen werden (Kanten�berdeckung). Der C1-Test setzt den C0-Test voraus bzw. schlie�t diesen ein und eignet sich ebenfalls f�r den Komponententest. Ziel des C1-Tests ist das Aufdecken von Kontrollflussfehlern.}
%\abbrev{Datenbank}{(\textit{DB} oder \textit{DataBase}): System zur Speicherung und Abfrage von Daten.}
\abbrev{Delegationskonnektor}{Ein Delegationskonnektor (delegation connector) zeigt die Verbindung zwischen dem "`�u�eren"' Verhalten der Komponente mit der internen Struktur (Parts), die das Verhalten realisiert. Diese Verbindungen werden stets �ber die Ports realisiert, wobei ein Delegationskonnektor immer zwischen dem Port der Komponente und dem Port des Parts realisiert wird.}
\abbrev{Drag-and-drop}{Eine Interaktionsmethode, bei welcher der Benutzer ein grafisches Objekt �ber die Bildschirmfl�che ziehen und am gew�nscht Ort fallen lassen kann und gegebenfalls mit einem darunterliegenden Objekt interagieren lassen kann, z. B. das Ziehen einer Datei in den Papierkorb, um sie zu l�schen.}
%\abbrev{EDV}{Elektronische Datenverarbeitung}
\abbrev{Exception}{Ausnahmefehler der w�hrend der Ausf�hrung des Programmes (Laufzeitfehler; Stromausfall etc.) auftritt.}
%\abbrev{Fenster}{(Ansicht): Ein Teil der grafischen Benutzer\-ober\-fl�\-che. Ein Fenster ist ein Ausschnitt aus dem Bildschirm, in dem separat verschiedene Funktionen ablaufen k�nnen.}
\abbrev{Fokus}{Bezeichnet das aktuelle Objekt, an oder in dem eine Manipulation durch den Benutzer durchgef�hrt werden kann, zum Beispiel ein aktives Fenster oder eine ausgew�hltes Zeichenobjekt.}
\abbrev{Funktionale Anforderungen}{Anforderungen, denen das Programm zur Erf�llung der von ihm zu leistenden Funktionalit�ten gen�gen muss.}
\abbrev{GUI}{\emph{Graphical User Interface}: Die grafische Benutzeroberfl�che; Schnittstelle zwischen dem Benutzer und der Funktionalit�t des Systems.}
%\abbrev{Hardware}{Die physischen Ger�te eines Computers (zum Beispiel Drucker, Monitor usw.).}
%\abbrev{Input}{Eingegebene Daten (Text, Grafik, usw)}
\abbrev{Intuitive Bedienung}{Das Programm ist ohne Vorkenntnisse bedienbar - man kann die Wirkung oder Vorgehensweise erahnen.}
\abbrev{Klasse}{Eine abstrakte Einheit eines Programms; besteht aus gleichartigen Objekten.}
\abbrev{Komponente}{In Ride.NET muss eine Komponenten (insbesondere COM-Komponente) in Form einer Visual Studio Projektdatei (.proj) vorliegen, damit sie als Quellcode importiert werden kann. Daher sind Komponenten auf Quellcodeniveau f�r den Palladio.Editor mit Projektdateien gleichzusetzen, die zus�tzlich verwendete Schnittstellen enthalten k�nnen. Eine Projektdatei kann auch mehrere Komponenten enthalten.}
\abbrev{Kompositionskonnektor}{Ein Kompositionskonnektor (assembly connector) ist die Verbindung zwischen angebotenen und benutzten Interfaces bzw. Ports. Somit wird darauf hingewiesen, dass eine Komponente Services bereitstellt, die von anderen gebraucht werden.}
\abbrev{Konsistenz}{auch Systemkonsistenz: Grad der �bereinstimmung zwischen Me�werten von Testeinzelleistungen; Softwarekorrektheit. Im Rahmen von Ride.NET: �bereinstimmung von Datenmodell und Sicht. Konsistenz ist dann hergestellt, wenn alle Plugins in sich auf den gleichen aktuellen Zustand des Datenmodells beziehen.}
%\abbrev{Konsistenzbedingungen}{Bedingungen, die zur Gew�hrleistung der Korrektheit von Software erf�llt sein m�ssen.}
\abbrev{Men�}{Teil der grafischen Benutzeroberfl�che; Befehle werden in Klartext auf dem Bildschirm ausgegeben und k�nnen �ber die Tastatur oder die Maus ausgew�hlt werden.}
\abbrev{Online-Hilfe}{ Ein Teil des Programms, das Funktionen des Programms w�hrend dessen Ablauf erkl�rt.}
%\abbrev{Plug-and-play}{Nicht zu verwechseln mit \textit{Drag-and-drop}, eine M�g\-lich\-keit, Hardwarekomponenten einfach einzubauen oder anzuschliessen um sofort damit arbeiten zu k�nnen.}
%\abbrev{PG}{Abk�rzung f�r Projektgruppe.}
\abbrev{Plugin}{Eine Programm, das in ein bestehendes Programm integriert und gew�hnlich nur in dessen Kontext ausf�hrbar ist, und so zus�tzliche Funktionen bereitstellen kann.}
\abbrev{Projekt}{siehe \textit{Ride.NET-Projekt}}
\abbrev{Projektkonfiguration}{Die Konfiguration eines Projektes in Ride.NET sind alle projektspezifischen Einstellungen. Diese umfassen insbesondere alle in einem Projekt benutzten Werkzeuge und deren Einstellung, alle darstellungsspezifischen Einstellungen (z.B. Farbgebung, Schrift etc.), alle umgebungsspezifischen Einstellungen (z.B. ge�ffnete Fenster, Darstellungssichten etc.) und alle zu einem Projekt zugeh�rigen Dateien und Informationen.)}
\abbrev{Protokollautomat}{Ein (endlicher) Automat, der das Protokoll einer Schnittstelle beschreibt.}
\abbrev{Reaktionszeit}{Im Zusammenhang mit der Benutzerschnittstelle stellt die Reaktionszeit eine nichtfunktionale Anforderung dar, die angibt, nach welcher Zeit ein Benutzer Reaktionen vom System auf seine Eingaben signalisiert bekommt. Werden beispielsweise umfangreichere Berechnungen angesto�en, so w�re die Zeit bis zum Erscheinen eines Fortschrittsbalkens die Reaktionszeit.}
\abbrev{Ressourcen}{Der Begriff Ressourcen wird in diesem Projekt synonym mit dem Wort Ressourcenbeschreibung verwendet.} 
\abbrev{Ride.NET-Projekt}{Ein Ride.NET-Projekt ist eine Sammlung aller Dateien und Informationen, bestehend aus einem Komponentenmodell, der zugeh�rigen VS-Solution-Datei, sowie aller zus�tzlichen Informationen und Konfigurationen, die zur Modellierung des o.g. Komponentenmodells in Ride.NET verwendet werden.}
\abbrev{SEFF}{\textit{Service EFFect Automat}: Ein endlicher Automat, der die �u�eren Auswirkungen des Aufrufes einer von der Komponenten angebotenen Schnittstelle beschreibt.}
\abbrev{Schnittstelle}{(auch \textit{Interface}) F�r den Palladio.Editor im Rahmen von Ride.NET muss eine Schnittstelle auf Quellcodeniveau in Form einer einzelnen Visual Studio Projektdatei (.proj) vorliegen, damit diese importiert werden kann. Dabei bezeichnet eine Schnittstelle einen definierten �bergabepunkt zur Kommunikation zweier Komponenten. Schnittstellen k�nnen ebenfalls mit implementierenden Komponenten zusammen in einer Visual Studio Projektdatei vorliegen.}
\abbrev{Code-Skeleton}{Programm-Grundger�st, das nur aus den elementarsten Programmteilen besteht, wie Klassen, Methoden oder Schleifen.}
%\abbrev{Software}{Programme, die auf einem Computer ausgef�hrt werden k�nnen.}
\abbrev{Softwaresystem}{System, dessen Systemkomponenten und -elemente aus Software bestehen; S. sind Produkte von Softwareprojekten.}
\abbrev{Quellcode}{der Programmquelltext, in dem das Programm f�r eine bestimmte Programmiersprache lesbar entworfen worden ist}
\abbrev{Shortcut}{Ein Shortcut ist eine Tastenkombination, mit der man eine Programmfunktion aufrufen kann.}
\abbrev{Systemebene}{Umgebung innerhalb eines Systems, die sich von anderen (Sub-)Umgebungen desselben Systems abgrenzt.}
\abbrev{Tab}{Karteireiter}
\abbrev{Tool}{Software-Werkzeug, welches die einfachere Bearbeitung komplexer Arbeitsvorg�nge erm�glicht; oft auf visuellem Weg.}
\abbrev{Toolbox}{Werkzeugkasten: Eine M�glichkeit, Werkzeuge und Funktionen auszuw�hlen, ohne eine Men�struktur zu benutzen}
\abbrev{UML}{\textit{Unified Modelling Language}: Grafische Sprache zur Beschreibung von Objekten eines Programms und deren Interaktionen.}
\abbrev{Redo}{Wird h�ufig bei Editoren eingesetzt. Mit Redo k�nnen r�ckg�ngig gemachte Ver�nderungen schrittweise wiederhergestellt werden.}
\abbrev{Undo}{Wird h�ufig bei Editoren eingesetzt. Mit Undo k�n\-nen Ver�nderungen schrittweise r�ckg�ngig gemacht.}
\abbrev{Validierung}{Validierungen werden bei Ride.NET auf dem Komponentenmodell und m�glichen spezifizierten Zusatzattributen ausgef�hrt. Sie bestimmen, ob ein gegebenes Modell einer bestimmte Definition von Validit�t gen�gt. So kann beispielsweise auf Schnittstellen eine Untermenge von einer Komponente angebotener Dienste als valide gelten, auch wenn in einem anderen Kontext eine andere Definition verwendet wird. Hier ist zu unterscheiden von \textit{Konsistenz}.}
\abbrev{VS}{Abk�rzung f�r Microsofts Visual Studio. In der Version 2003 und 2005 Software Entwicklungsumgebung f�r Programmiersprachen wie C\# und J\#.}

		
	% Index
	%\newpage
	%\lhead{Index} \chead{} \rhead{\thepage}
	%\addcontentsline{toc}{section}{Index}
	%\begin{theindex}

  \item Abstrakte Basisklassen, \hyperpage{1}

  \indexspace

\end{theindex}

	
	
	%Changelog
	%\newpage	
	%\addcontentsline{toc}{section}{Change Log}
  %	\lhead{Change Log} \chead{} \rhead{}
	%\input{changelog.tex}
	
	% QS-Bemerkungen
	%\newpage	
	%\addcontentsline{toc}{section}{Qualit�tssicherung}
  %	\lhead{Qualit�tssicherung} \chead{} \rhead{}
	%% $Log$
% Revision 1.1  2006/01/03 10:37:55  kelsaka
% - move to folder proposal
%
% Revision 1.1  2005/12/30 15:16:19  kelsaka
% initial creation
%

\newpage
\section*{Qualit�tssicherung - Anmerkungen}
\begin{tabular} {|p{1,5cm}|p{3cm}|p{5,5cm}|p{5cm}|}
	\hline
	\textbf{Datum}&\textbf{gepr�ft von}&\textbf{Art der �nderung} & \textbf{Anmerkung}\\
	\hline
\end{tabular}
\newpage


	
\end{document}
