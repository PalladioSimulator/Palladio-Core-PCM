\documentclass[a4paper, 12pt]{article}
\usepackage[OT1]{fontenc}
\usepackage{ngerman}
\usepackage[latin1]{inputenc}
\usepackage{graphics}
\usepackage{fancyhdr}
\usepackage{epsfig}
% nur f�r linux's pdflatex
%\usepackage{pdfpages}
%active table of contents but without borders
\usepackage[colorlinks=false,citebordercolor=111,menubordercolor=111,linkbordercolor=111]{hyperref}

%definiere absatz nach deutschem Standard (keine Einrckung, dafr gr�serer Zeilenabstand):
\setlength{\parindent}{0pt} \setlength{\parskip}{5pt plus 2pt minus 1pt}


\begin{document}

% Kopfzeile mit Kapitel
\renewcommand{\sectionmark}[1]{\markleft{#1}{}}
\pagestyle{fancyplain}
\rhead{\footnotesize\thepage}
\lhead{\footnotesize\leftmark}
\cfoot{}

%Titelseite
\begin{titlepage}
\title{\includegraphics[width=7cm]{uniol.png}\\ \vspace{2cm}				
			\textbf{Palladio Componentmodel\\}
			{ \large Entwurfsbeschreibung\vspace{2cm}}}
			
			\author{Marko Hoyer\\ \textit{Marko.Hoyer@informatik.uni-oldenburg.de}}
\maketitle

\vspace{3cm}

\thispagestyle{empty}
\end{titlepage}

%Inhaltsverzeichnis
\tableofcontents
\thispagestyle{empty}
\newpage

%Einleitung
\section{Einleitung}
\label{sec:einleitung}
Das hier vorgestellte Individuelle Projekt befasst sich mit der Simulation des Laufzeitverhaltens von Komponentennetzwerken. Hierbei steht die Analyse eines Netzwerkes vor dessen Implementierung im Vordergrund. 
Das jetzt folgende praktisches Beispiel soll im Vergleich zur mathematisch exakten Berechnung den Sinn der Simulation verdeutlichen.

\subsection{Simulationsumgebung vs. mathematische Berechnung}
\label{sec:einleitung:simvsmath}
Zur Entwicklung eines Systems sollen eine Reihe von Komponenten eingesetzt werden, welche in Kombination die Dienste des Systems implementieren. Die Komponenten sind hierbei durch ihre Konnektoren, ihre Dienste, einen inneren Kontrollfluss und ihre Laufzeiteigenschaften gekennzeichnet. Weiterhin ist die Verkn�pfung der Komponenten untereinander durch den Entwickler des Systems vorgegeben.
\par
Ziel der Simulation soll die Auswertung der Dienste des Gesamtsystems sein. Hierbei spielt neben der Antwortzeit der einzelnen Dienste des Systems die Identifizierung von 'Flaschenh�lsen' innerhalb des Komponentennetzwerks eine Rolle.\\
Besteht das System ausschlie�lich aus linear zusammenh�ngenden Komponenten, deren Dienste der Reihe nach von einer ankommenden Anfrage durchlaufen werden, so gestaltet sich die Analyse des Systems recht einfach. Problemf�lle lassen sich anhand der Einzelzeiten identifizieren und die gesamte Antwortzeit kann durch Addition der Einzelzeiten relativ leicht ermittelt werden.\\
Beinhalteten die Komponenten innerhalb des Systems jedoch Verzweigungen, so m�ssen alle sich ergebenen Pfade einzeln berechnet und mit einer bestimmten Gewichtung gewertet werden.\\
Weiterhin ergibt es sich in Systemen h�ufig, das bestimmte Dienste einer Komponente von mehren Diensten des Gesamtsystems ben�tigt werden. Ein Beispiel hierf�r sind Dienste, die Daten aus einer Datenbank auslesen. Hierbei geht die Analyse des Systems �ber die Pfade hinaus. Es m�ssen nun die Antwortzeiten der Dienste der Komponenten dynamisch auf die Anzahl zu einem Zeitpunkt ankommender Anfragen angepasst werden.\\
�bersteigt die mathematisch exakte Analyse des Systems bereits hier die Grenzen des sinnvoll machbaren, so erscheint die exakte Berechnung bei der Verteilung der einzelnen Komponenten auf verschiedene Prozessoren als unm�glich.
\par
An dieser Stelle kann die Simulation ansetzen. Es werden nun nicht mehr die mathematisch exakten Gegebenheiten berechnet sondern anhand der Simulation eines Modells mit einer bestimmten Ungenauigkeit ermittelt. Weiterhin lassen sich bei der Simulation 'Flaschenh�lse' identifizieren, die bei der mathematischen Berechnung nur schwer zu ermitteln sind. Hierzu kann beispielsweise einfach das Zeitverhalten einer Anfrage Dienst f�r Dienst aufgezeichnet und hinterher ausgewertet werden. Bild \ref{pic:simul} zeigt schematisch eine solche Simulation.

\begin{figure}[ht]
\label{pic:simul}
\begin{center}
\fbox{\includegraphics[width=13cm]{../res/simul.jpg}}
\caption{Schematische Darstellung der Simulation}
\end{center}
\end{figure}

%Architektur
\section{Architektur}

\begin{figure}[ht]
 \centering \includegraphics[scale=0.8]{arch.png}
 \caption{Architektur des Komponentenmodells}
 \label{fig:arch}
\end{figure}

%kern
\section{Datenhaltung im Modellkern}

Der Modellkern, dessen Hauptaufgabe die Verwaltung der Daten darstellt, bildet den wichtigsten Bestandteil des Komponentenmodells mit den im Folgenden kurz erl�uterten Anforderungen.

\begin{itemize}
\item \textbf{Speicherung}\\
Der Modellkern muss in der Lage sein, alle Daten zur Laufzeit des nutzenden Programms zu speichern. Zu den Daten geh�ren die Entit�ten des Komponentenmodells mit ihren Attributen. Die Struktur der Attribute beschr�nkt sich hierbei nicht auschlie�lich auf Standartdatentypen. Es m�ssen beliebige z.T. zur Entwurfszeit unbekannte Datenstrukturen speicherbar sein. Weiterhin m�ssen die Beziehungen zwischen den Entit�ten (z.B. Komponente \emph{A} enth�lt Komponente \emph{B}) festgehalten werden. 

\item \textbf{Konsistenzpr�fung}\\
Wie eingangs in der Architekturbescheibung erl�utert besteht die Aufgabe des Modellkerns nicht in der Implementierung der Konsistenzpr�fung des theoretischen Modells. Somit sollte im Idealfall prinzipiell erst einmal alles abspeicherbar sein. Da die Umsetzung dieser Anforderung viele ungenutzte und zu Lasten der Komplexit�t fallende M�glichkeiten bietet, ist die Nutzung von Wissen �ber das theoretische Modell bei der Konzeption der Datenhaltung sinnvoll einzubringen. Verst��e gegen die sich hierraus ergebenen Beschr�nkungen sind dann jedoch durch den Modellkern abzufangen und entsprechend zu behandeln. Soll beispielsweise entsprechend dem o.g. Beispiel die Komponente \emph{B} der Komponente \emph{A} hinzugef�gt werden, so ist vom Modell diese Beziegung zu speichern. Setzt die gew�hlte Speicherstruktur hierbei das vorhandensein von Komponente A vorraus, so ist das durch den Modellkern sicherzustellen. Dieser kann dann entweder die Speicherung ablehnen oder selbst�ndig eine Komponente A erzeugen.

\item \textbf{Zugriffsmethoden}\\
Die dritte Anforderung an den Modellkern stellen die Zugriffsmethoden dar. Da in die Datenhaltung, wie oben erl�uert, nicht das vollst�ndige Wissen �ber das theoretische Modell zu implementieren ist, k�nnen keine hierauf zugeschnittenen Zugriffsmethoden zur Verf�gung gestellt werden. Es ist also eine Schnittstelle zu schaffen, die flexiblen Zugriff auf alle gespeicherten Daten bereitstellt. Bestehen im Modell der Datenhaltung bereits Beziehungen zwischen den Daten, so bietet sich deren Nutzung beim Zugriff an.
Weiterhin wichtig ist sowohl bei den Zugriffsmethoden als auch bei der Speicherung die Geschwindigkeit. Dieser Teil des Modells bildet, wie bereits erl�utert, die Datenhaltung f�r die laufende Anwendung. Sorgt die Arbeit auf dem Modell f�r zu hohe Latenz, so leidet die Nutzbarkeit der Anwendung hierrunter stark.

\end{itemize}

Zur Umsetzung des Modellkerns kommen eine Reihe von Strategien in Frage, von denen drei im Folgenden gegeneinander abgegrenzt werden. Abschlie�end folgt in diesem Kapitel die ausf�hrliche Vorstellung der in dieser Version des Komponentemodells implementierten Variante.

Die erste Strategie bedient sich ausschlie�lich objektorientierter Konzepte. Hierbei werden die Entit�ten durch Klasseninstanzen und Beziehungen zwischen diesen durch Referenzen auf andere Instanzen modelliert. Vorteile dieser Variante ergeben sich aus guter Modellierbarkeit von Spezialisierung, problemloser Speicherung von Attributen unbekannten Typs und hoher Geschwindigkeit. Erfahrungen haben gezeigt, dass sich bei der Umsetzung dieses Konzeptes Probleme hinsichtlich Wartbarkeit und Erweiterbarkeit ergeben, die sich auf die starke Abh�ngigkeit der Klassen untereinander zur�ckf�hren lassen. Ebenfalls schwierig zu modellieren sind auf diese Art zirkul�re Abh�nigkeiten.

Der zweite Ansatz bedient sich einer relationalen oder einer objektrelationalen Datenbank. Die Entit�ten werden hierbei in entsprechenden Tabellen der Datenbank gespeichert. Die Beziehungen zwischen den Entit�ten lassen sich in der Datenbank entsprechend als Beziehungen zwischen den Tabellen modellieren. Details zum Entwurf solcher Datenbankschemata und deren Nutzung sei an dieser Stelle auf entsprechende Literatur (z.b. \cite{lit:db}) verwiesen. 

\begin{figure}[ht]
 \centering \includegraphics[scale=0.53]{dataset.png}
 \caption{.NET Dataset des Modellkerns}
 \label{fig:dataset}
\end{figure}

%environment
\section{Instanzierung des Modells}

Das Komponentenmodell ist durch die im Namensraum \verb+Palladio.ComponentModel+ definierte Klasse \verb+ComponentModelEnvironment+ gekapselt. Diese l��t sich unter Verwendung des parameterlosen Konstruktors instanzieren. Mehrfache Instanzen des Komponentmodells sind problemlos m�glich, da auf die Verwendung von Klassen gem�� dem Singleton-Pattern basierend auf statischen Klassenvariablen \cite{lit:gof} verzichtet wurde. Beim Entwurf von Erweiterungen, die direkt in das Komponentenmodell einflie�en, ist dieses Konzept beizubehalten, um Kompatibelit�tsprobleme zu vermeiden. Konzepte zur Synchronisation zweier Modelle sind in der aktuellen Version des Komponentmodells nicht vorgesehen. Bestehen Anforderungen dieser Art, so sind diese unter Verwendung der Benachrichtigungsmechanismen in Verbindung mit der Anfrage und Builder-Schicht zu realieren.

Nach der Instanzierung steht ein leeres Modell zur Verf�gung. Wahlweise kann unter Verwendung der Builder (vgl. Kapitel \ref{sec:builder}) ein neues Modell erstellt oder ein persistent gespeichertes geladen werden. Alle Schnittstellen zu den jeweiligen Schichten sind �ber die das Modell kapselnde Klasse \verb+ComponentModelEnvironment+ erreichbar.

%Builder
\section{Aufbau eines neuen Modells}

%Events
\section{Benachrichtigung bei �nderungen im Modell}
\label{sec:events}

%Query
\section{Suchanfragen an das Modell}

\subsection{Allgemeine Anfragen}

%Navigation
\input{navigation.tex}

%Equals
\subsection{Vergleichbarkeit zwischen Bestandteilen des Modells} 

%Serialisierung
\section{Persistente Speicherung des Modells}

%andere
\input{andere.tex}

%Literaturverzeichnis
\addcontentsline{toc}{section}{Literaturverzeichnis}%
\bibliographystyle{geralpha}
\bibliography{cm_entwurf}


\end{document}
