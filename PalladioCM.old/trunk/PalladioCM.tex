% Version history:
% $Log$
% Revision 1.8  2005/10/12 10:01:48  sliver
% no message
%
% Revision 1.7  2005/08/17 15:19:44  sliver
% added Outline.Tex. Its purpose is the creation of a thread (roter Faden) for the paper.
% The file cantains a slightly different structure of the original paper.
% For each section an (often incomplete but nevertheless useful) outline was added.
% Its a basis of discussion. Refinement follows.
%
% Revision 1.6  2005/08/17 11:20:52  sbecker
% Jens und Steffens Gliederungserweiterung
%
% Revision 1.5  2005/08/15 15:53:50  sliver
% no message
%
% Revision 1.4  2005/08/15 12:13:40  sbecker
% Review done by Jens, most changes committed
%
% Revision 1.3  2005/08/11 11:31:38  sbecker
% Added three tier architecture example in section examples
%
% Revision 1.2  2005/07/14 12:40:24  sbecker
% Fixed typos in the intro
% Added parts and authors of section 2
%
% Revision 1.1  2005/07/02 15:33:14  rr
% erste Fassung
%
\documentclass{llncs}

% Package for EPS-graphics
\usepackage{graphicx}

\usepackage{times}
\usepackage{t1enc}

\usepackage{epsfig}
\usepackage{subfigure}
\usepackage{fancyvrb}

\usepackage{float}
\floatstyle{plain}
\newfloat{codesample}{htbp}{loco}
\floatname{codesample}{Code example}


\begin{document}

% Paper title
\title{
	The Palladio Component Meta Model:\\ Towards an Engineering Approach\\ to Software Architecture Design
}

\author {Ralf H. Reussner \and Steffen Becker \and Jens Happe \and Klaus Krogmann \\
	\email{{ralf.reussner|steffen.becker|\\ jens.happe|klaus.krogmann}@informatik.uni-oldenburg.de} 
}
\institute{
	Software Engineering Group, Department of Computing Science\\University of Oldenburg,  Germany \\
}
\maketitle			

\begin{abstract}
  
\end{abstract}

\section{Introduction}
%contribution
%organisation

\section{

\section{Conclusions}


\bibliographystyle{splncs}
\bibliography{palladio}


\end{document}
