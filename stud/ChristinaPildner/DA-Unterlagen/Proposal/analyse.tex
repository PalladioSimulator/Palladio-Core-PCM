%% analyse.tex
%% $Id: analyse.tex 5 2005-10-10 20:55:48Z bless $

\chapter{Analyse}
\label{ch:Analyse}
%% ==============================
In diesem Kapitel sollten zun�chst das zu l�sende Problem
sowie die Anforderungen und die Randbedingungen 
einer L�sung beschrieben werden (also nochmal
eine pr�zisierte Aufgabenstellung).

Dann folgt �blicherweise ein �berblick �ber bereits existierende
L�sungen bzw. Ans�tze, die meistens andere Voraussetzungen bzw.
Randbedingungen annehmen.

Bla fasel\ldots

%% ==============================
\section{Anforderungen}
%% ==============================
\label{ch:Analyse:sec:Anforderungen}
Anforderungen und Randbedingungen \ldots

%% ==============================
\section{Existierende L�sungsans�tze}
%% ==============================
\label{ch:Analyse:sec:RelatedWork}

Hier kommt eine ausf�hrliche Diskussion
von "`Related Work"'.

Bla fasel\ldots

%% ==============================
\section{Weiterer Abschnitt}
%% ==============================
\label{ch:Analyse:sec:Abschnitt}

Bla fasel\ldots

%% ==============================
\section{Zusammenfassung}
%% ==============================
\label{ch:Analyse:sec:zusammenfassung}

Am Ende sollten ggf. die wichtigsten Ergebnisse nochmal in \emph{einem}
kurzen Absatz zusammengefasst werden.

%%% Local Variables: 
%%% mode: latex
%%% TeX-master: "diplarb"
%%% End: 
