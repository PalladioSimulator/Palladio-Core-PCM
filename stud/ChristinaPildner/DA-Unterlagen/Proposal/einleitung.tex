%% Einleitung.tex


\chapter{Einleitung}  \label{ch: Einleitung}

Dieser Proposal beschreibt die Planungen zur Umsetzung der Diplomarbeit "`Modellbasierte Analyse von Sicherheitsschwachstellen in objektorientierten Modulen"'. Diese \textit{Einleitung} befasst sich mit der Motivation und die Ziele der Diplomarbeit. Der n�chsten Kapitel \textit{Konzeption} 

Um die Lesbarkeit zu f�rdern, wird des weiter der Begriff "`objektorientiert"' als "`OO"' gekennzeichnet. 

\section{Hintergrund und Motivation} \label{ch: Einleitung: sec: Hintergrund und Motivation}

Enge Interaktionen in kritischen Anwendungen zwischen Dienste und Bibliotheken von verschiedener Herkunft finden immer h�ufiger statt, entweder in gro�en statischen Systemen in den viele Komponente wiederverwendet werden, oder kleine, dynamischen Umgebungen wie der OSGi-Platform.  Desweiteren erfordern steigende Produktivit�tsbed�rfnisse, dass Testen und Analyse kosteng�nstiger werden, und deswegen automatisiert werden sollen. 

\section{Ziele}

Ein Teil der Diplomarbeit besteht aus der Definition und Klassifikation der Sicherheitsschwachstellen in OO-Modulen in Java. Dieses erfordert eine vorgehende Modellbildung, damit die Sicherheitskriterien in kompatible Formalismen ausgedruckt werden k�nnen. Eine Sicherheitsschwachstelle wird durch die Problemmuster identifizierbar, die sie verursacht. Die Definition einer Sicherheitsschwachstelle wird durch vorhandene Modelle dargestellt. Unter verschiedene M�glichkeiten der Analyse und Darstellung,  wird PQL (Programm Query Language) \ und SISSy( Structural Inprovement of Software System) in Betracht gezogen. 
 
PQL ist eine Anfragespache, die dem Programmierrer


Ein weiteres Teil der Diplomarbeit besch�ftigt sich mit der Erstellung eines Tools, das die Erkenntnisse und des ersten Teils implementiert. Dieses Tool soll Programmierern das Auffinden und Verstehen der Vulnerabilit�ten aus eigenem Code erleichtern, indem es Vulnerabilit�ten in Code entdeckt, dem Programmierer darstellt und Informationen dazu aufblendet. 

  
 Aufgabenstellung:
-	Definieren von Sicherheitsschwachstellen in OO-Komponenten, 
o	Schwachstellen sollen Problemmuster verursachen, die identifizierbar sind 
o	Erfordert eine vorgehende Modelbildung, damit die Sicherheitskriterien in kompatiblen Formalismen ausgedruckt werden k�nnen. 
M�gliche Aufgaben
-	Definition und Formalisierung der Sicherheitsschw�chen in OO-Modulen
-	Identifizierung notwendiger Elemente zu Quellcode Analyse
-	Entwicklung einer prototypischen Analyse
-	Validierung der Analyse anhand von OO-Modulen und Bibliotheken
Das Hauptziel der Diplomarbeit ist, Sicherheitsschwachstellen in 

Das praktische Teil der Diplomarbeit ist, ein Tool f�r Statische Analyse zu entwickeln, die dem Programmierer unterst�tzen soll, ein sicheren Quellcode in Java zu erstellen. 
-	Sicht des Benutzter 
o	Was macht das Tool
"	Statische Analyse des Quellcodes 
"	Abblenden von Informationen und Hinweise w�hrend der Programmierung 
"	Info basierend auf dem Catalog von PP (Siehe Diss)
o	Welche Eingenschaften soll er haben?
"	Bediehnbarkeit
"	Einfach zu verstehen
"	Einfach zu bedienen
-	Sicht der Programmierer
o	Eigenschaften
"	Einfach zu erweitern (Modularisierung+Dokumentation)
"	Einf�gen von weiteren Begriffen
o	Basierend auf 
"	SISSy?
"	
1.3	Artefakte
W�hrend der sechs monatigen Zeit der Durchf�hrung sollen folgende Artefakte entstehen
"	Literatur�berblick �ber bereits vorhandene �hnliche Forschungsthemen, Tools, Verfahren und Konzepte.
"	Softwareprogram, die erforschte Konzepte implementiert
"	Ausarbeitung des Forschungsthemas.
Desweiteren sind wissenschaftlichen Arbeiten und Ver�ffentlichungen erw�nscht. 
2.	Grundlagen

3.	Konzeption 
-	Formale Beschreibung der Vulnerabilit�t
-	Katalogen 
-	Analyse erw�hnen (Statische und 
3.1	Erwartete Qualit�tsmerkmale 
-Vertraulichkeit
-Verf�gbarkeit
-korrekte Zugangskontrolle
3.2	Architektur
4.	Durchf�hrung
4.1	Forschung und Analyse
4.2	Entwicklungsumgebung 
F�r die Erstellung der Implementierung wird Eclipse als Entwiklungsplattform verwendet. Als Programmiersprache wird Java verwendet.
Es werden weitere Eclipse-Plugins und 
4.3	Organisatorisches
Erstgutachter: Prof. Dr. Ralf H. Reussner
Zweitgutacher: 
Betreuer: Dr. Pierre Parrend
Die Diplomarbeit findet am Institut f�r  Programmstrukturen und Datenorganisation (IPD), Lehrstuhl f�r Software Design und Qualit�t (SDQ), in Kollaboration mit Forschungszentrum Informatik (FZI).  
4.4	Zeitplan
-	Meilensteine
Gem�� der Pr�fungsordnung der Universit�t Karlsruhe (TH) f�r 
den Diplomstudiengang Informatik vom  betr�gt die Bearbeitungszeit einer Diplomarbeit an der Uni Karlsruhe von der Anmeldung bis zum Abschluss sechs Monate. Ein grober Zeitplan mit einigen Meilensteinen und bis dahin erstellten Artefakten ist in der folgenden Gantt-Abbildung dargestellt. Das Diagramm stellt nur den Rahmen der Diplomarbeit dar, im Laufe der Zeit soll der Ablauf verfeinert werden.
W�hrend der Bearbeitung finden regelm��ige Treffen zwischen Pr�fling und Betreuer statt, mindestens einmal in der Woche. Desweiteren besucht der Pr�fling die regelm��igen Treffen der Diplomanten am SDQ und FZI. 



%%% Local Variables: 
%%% mode: latex
%%% TeX-master: "diplarb"
%%% End: 
